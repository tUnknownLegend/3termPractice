
\documentclass[12pt, a4paper]{article}

\usepackage[utf8]{inputenc}
\usepackage[T1]{fontenc}
\usepackage[russian]{babel}
\usepackage[oglav,spisok,boldsect,eqwhole,figwhole,hyperref,hyperprint,remarks,greekit]{./style/fn2kursstyle}
\graphicspath{{./style/}{./figures/}}

\usepackage{multirow}
\usepackage{supertabular}
\usepackage{multicol}
% Параметры титульного листа
\title{Математика древней и средневековой Индии}
\author{В.\,Г.~Пиневич}
\supervisor{О.\,В.~Пугачев}
\group{ФН2-81Б}
\date{2024}

% Переопределение команды \vec, чтобы векторы печатались полужирным курсивом
\renewcommand{\vec}[1]{\text{\mathversion{bold}${#1}$}}%{\bi{#1}}
\newcommand\thh[1]{\text{\mathversion{bold}${#1}$}}
%Переопределение команды нумерации перечней: точки заменяются на скобки
\renewcommand{\labelenumi}{\theenumi)}
\begin{document}

\maketitle

\tableofcontents



\newpage

\section-{Введение}
Математика древней и средневековой Индии представляет собой уникальную систему знаний и методов, которые оказали значительное влияние на развитие математики в мировой истории. Индия считается одним из родин древней математики, где были разработаны многие математические концепции и методы, которые были впоследствии усвоены и применены математиками других культур.

\section-{Позиционная система счисления}
Одной из важнейших отличительных особенностей математики древней Индии является использование позиционной системы счисления. Позиционная система счисления --- система счисления, в которой значение каждого числового знака (цифры) в записи числа зависит от его позиции (разряда) относительно десятичного разделителя.
Каждая позиция в числе имела свое значение, которое зависело от ее положения. Например, число 532 представляло собой 5 сотен, 3 десятки и 2 единицы. Это позволяло легко представлять и работать с большими числами.
Позиционная система счисления в древней Индии также использовала специальные символы для обозначения разрядов, такие как <<шунья>> (нуль), который был введен в VI веке н.э. Этот символ был ключевым в развитии математики и науки в древней Индии, а также в последующем развитии десятичной системы счисления.
Благодаря этому индийцы смогли применять сложные алгоритмы вычислений, что позволило им делать точные расчеты и решать сложные задачи. Позиционная система в древней Индии основывалась на десятичной системе счисления и использовала символы для представления чисел от 0 до 9. Эта система была развита в древней Индии в VIII веке н.э. и стала основой для многих современных систем счисления. 
Первые позиционные изображения были известны индийским астрономам примерно за 500 лет до Бхаскары I работы. Однако эти числа до Бхаскары I записывались не цифрами, а словами или аллегориями и были организованы в стихи. Например, число 1 было дано как луна, поскольку оно существует только один раз; число 2 было представлено крыльями, близнецами или глазами, поскольку они всегда встречаются парами; число 5 было дано 5 чувствами. Подобно нашей нынешней десятичной системе , эти слова были выровнены таким образом, что каждое число присваивает коэффициент степени десяти, соответствующий его положению, только в обратном порядке: более высокие степени были справа от нижних.
Древнейшая известная запись позиционной десятичной системы обнаружена в Индии в 595 году. В Европе о ней рассказал среднеазиатский математик аль-Хорезми, а поскольку его труд был написан на арабском, то за индийской нумерацией в Европе закрепилось иное название~--~<<арабская>>. 

\section-{Брахмагупта}
Одним из наиболее известных индийских математиков был Брахмагупта, который жил в VII веке. 

В своей работе Брахма-спхута-сиддханта Брахмагупта дал определение нуля как результат вычитания из числа самого числа. Он одним из первых установил правила арифметических операций над положительными и отрицательными числами и нулём, рассматривая при этом положительные числа как имущество, а отрицательные числа как долг. Далее Брахмагупта пытался расширить арифметику дав определение деления на ноль. Согласно Брахмагупте, деление нуля на ноль есть ноль,
деление положительного или отрицательного числа на ноль есть дробь с нулём в знаменателе,
деление нуля на положительное или отрицательное число есть ноль.
Брахмагупта предложил три метода умножения многозначных чисел в столбик (основной и два упрощённых), которые близки к тем, что используются в настоящее время. Основной метод Брахмагупта назвал <<гомутрика>>, что в переводе Ифра означает <<как траектория мочи коровы>>.

Брахмагупта также предложил метод приближённого вычисления квадратного корня, эквивалентный итерационной формуле Ньютона, метод решения некоторых неопределённых квадратных уравнений вида $a x^2 + c = y^2$, метод решения неопределённых линейных уравнений вида $a x + c = b y$, используя метод последовательных дробей.

Он определил сумму квадратов и кубов первых n чисел через сумму первых n чисел, утверждая что <<Сумма квадратов есть сумма чисел, умноженная на удвоенное число шагов, увеличенное на единицу, и делённая на три. Сумма кубов есть квадрат суммы чисел до одного и того же числа>>.

Труды Брахмагупта стали основой для многих последующих открытий в математике.

\section-{Бхаскара II}
Еще одним выдающимся математиком Индии был Бхаскара II, живший в XII веке. Он изучал алгебру, тригонометрию и астрономию, и его труды оказали огромное влияние на развитие математики в мире. Бхаскара II разработал формулы для вычисления площадей и объемов различных геометрических фигур, а также ввел понятие нуля как цифры.

Некоторые из вкладов Бхаскары в математику включают следующее:
\begin{enumerate}
\item Доказательство теоремы Пифагора путем вычисления той же площади двумя разными способами с последующим сокращением членов чтобы получить $a + b = c$.
\item В Лилавати решения квадратичного, кубического и четвертого неопределенного равно поясняются пункты.
Решения неопределенных квадратных уравнений (типа $a x + b = y$).
\item Целочисленные решения линейных и квадратных неопределенных уравнений. Правила, которые он дает, те же самые, что были даны Ренессансом европейскими математиками 17 века
\item Циклическим методом Чакравалы для решения неопределенных уравнений форма $a x + b x + c = y$. Решение этого уравнения традиционно приписывалось Уильяму Браункеру в 1657 году, хотя его метод был сложнее, чем метод чакравалы.
\item Первый общий метод поиска решений задачи $x - ny = 1$ было дано Бхаскарой II.
\item Решения диофантовых уравнений второго порядка, например $61 x + 1 = y$. Это уравнение было поставлено в качестве проблемы в 1657 г. французским математиком Пьером де Ферма, но его решение не было известно в Европе до времен Эйлера в 18 век.
\item Решил квадратные уравнения с более чем одним неизвестным и нашел отрицательные и иррациональные решения.
\item Предварительная концепция бесконечно малого исчисления, наряду с заметным вкладом в интегральное исчисление.
\item Задуманное дифференциальное исчисление, после обнаружения приближения производная и дифференциальный коэффициент.
\item Заявленная теорема Ролля, частный случай одной из важнейших теорем анализа, Теорема о среднем значении. Следы общей теоремы о среднем значении также найдены в его работах.
\item Вычислял производные тригонометрических функций и формул.
\item В Сиддханта-Широмани Бхаскара разработал сферическую тригонометрию вместе с рядом других тригонометрических результатов.
\item Оценка числа $\Pi$.
\end{enumerate}

\section-{Ариабхата}
Ариабхата (или Арьябхата, Кусумапури, 476—550), выдающийся индийский математик и астроном. Долгое время его путали с учёным того же имени, жившим на IV века позднее; сейчас последнего называют Ариабхата II. Его деятельность открывает «золотой век» индийской математики.
В сочинении <<Ариабхатия>> в стихотворной форме изложены математические сведения, необходимые для астрономических вычислений. Здесь вводятся тригонометрические функции синус и косинус, вычисляется таблица синусов. Ариабхата нашёл весьма точное значение числа $\Pi$ (3,1416). Ариабхата приводит правила суммирования рядов треугольных чисел, натуральных квадратов и кубов; даёт исчерпывающее решение квадратного уравнения. В связи с проблемой повторяемости небесных движений Ариабхата рассматривает неопределённые уравнения первой степени с двумя целочисленными неизвестными и изобретает для их решения метод измельчения.
Из сочинений, написанных Ариабхатой, известны названия двух — «Ариабхатия» (499) и «Ариабхата-сиддханта», но сохранился текст лишь одного — «Ариабхатии». Оно состоит из четырех частей, изложенных в стихотворной форме в 123 шлоках (стихах): дашагитика (система чисел, астрономические константы и таблица синусов), ганитапада (математика), калакрийа (календарь, расчёты соединений планет и обращений по эпициклам), голапада (основы сферической астрономии и расчёты затмений).
Изложение Ариабхаты — краткое до чрезвычайности. По форме это стихотворный текст, содержащий основные правила, к которым дополнительно требуется устный комментарий учителя.
«Ариабхатия» оказала огромное влияние на всё последующее развитие математики и астрономии в Индии и положило начало новой научной традиции в этой стране. В середине VIII века трактат Ариабхаты перевёл на арабский язык багдадский астроном Абу’л-Хасан Ахвази (Abu’l-Hasan Ahwazi fl. 830), применявший «систему Ариабхаты» в своих расчётах.
В математической части трактата Ариабхата:
- описывает процесс извлечения квадратного и кубического корня в десятичной системе счисления;
- даёт формулы для площади круга и объёма сферы;
- приводит также приближённое значение для числа пи — отношения длины окружности к её диаметру ((4 + 100) × 8 + 62000)/20000 = 62832/20000 = 3,1416);
- приводит правило проверки результата с помощью девятки;
- рассматривает вычисление гипотенузы прямоугольного треугольника по данным катетам и некоторые другие расчётные формулы, основанные на теореме Фалеса и теореме Пифагора;
- даёт решение квадратного уравнения, возникающего в задаче на сложные проценты;
- приводит правила суммирования рядов треугольных, квадратных и кубических чисел.

\section-{Варахамихира}
Варахамихира, чьи работы датируются примерно V-VI веками н.э., был выдающимся индийским математиком, астрономом и астрологом. Его вклад в развитие математики был значительным.

Одно из наиболее известных произведений Варахамихиры - "Брихат-Самхита" ("Большая сборка"), состоящая из 105 плавно переходящих друг в друга глав, охватывающих широкий спектр тем, включая астрономию, астрологию, геометрию, арифметику и многие другие области.

В его работах были представлены методы вычисления сферических и геодезических величин, таких как длина дня и ночи, а также методы определения положения планет на небесной сфере. Он также описал различные методы для проведения астрономических наблюдений, включая использование инструментов для измерения углов и времени.

В области математики Варахамихира сделал значительные вклады в алгебру, теорию чисел и геометрию. Он разработал алгоритмы для решения линейных и квадратных уравнений, а также описал методы для вычисления квадратных корней и нахождения объемов геометрических фигур.

Кроме того, Варахамихира также известен своим вкладом в область астрологии, где он разработал методы прогнозирования погоды и предсказания событий на основе астрологических данных.

В общем, Варахамихира оказал огромное влияние на развитие математики и астрономии в Индии и во всем мире, его работы стали основой для многих последующих исследований и разработок.

\section-{Магавира}
Магавира, также известный как Магха, был индийским математиком и астрономом, который жил примерно в VIII веке н.э. Его вклад в развитие математики был значительным, особенно в области алгебры.

Его наиболее известным трудом является "Ганита-сарасанграха" («Собрание жемчужин алгебры»), который содержит ценные математические знания того времени. В этой работе Магавира представил множество методов для решения алгебраических уравнений, включая линейные, квадратные и даже биквадратные уравнения. Он также предложил методы для вычисления квадратных корней и кубических корней, а также для решения систем уравнений.

В "Ганита-сарасанграха" Магавира также описал различные математические теоремы, включая теорему о сумме арифметической прогрессии и теорему Пифагора. Его работы имели значительное влияние на развитие математики в Индии и в других частях мира.

Кроме того, Магавира также внес вклад в область астрономии, включая разработку методов для вычисления движения планет и звезд.

В целом, Магавира считается одним из великих индийских математиков и его работы продолжают оставаться важными источниками знаний для современных математиков.

\section-{Шридхара}
Шридхара (также известный как Шридхара Ачарья) был индийским математиком и астрономом, который жил в XII веке н.э. Его вклад в развитие математики был значительным, особенно в области алгебры.

Наиболее известным трудом Шридхары является "Патиганита" ("Совершенное вычисление"), который является комментарием к "Лилавати" - классическому труду Бхаскары II. "Патиганита" содержит обширные обсуждения алгебры, арифметики и геометрии.

В своем труде Шридхара представил методы решения широкого спектра математических задач, включая алгебраические уравнения, смешанные задачи, пропорции и дробные числа. Он также описал различные методы для вычисления площадей и объемов геометрических фигур, а также для нахождения корней уравнений.

Кроме того, Шридхара внес вклад в область астрономии, где его работы затрагивали темы, такие как планетарное движение и астрономические таблицы.

Шридхара оказал значительное влияние на развитие математики и астрономии в Индии, его труды продолжают быть объектом изучения и вдохновлять математиков в настоящее время.

\section-{Средневековье}
В средневековом периоде математика в Индии продолжала развиваться, хотя в это время страна потеряла свою ведущую роль в математической науке. XVI век был отмечен крупными открытиями в теории разложения в ряды, переоткрытыми в Европе 100—200 лет спустя. В том числе — ряды для синуса, косинуса и арксинуса. Поводом к их открытию послужило, видимо, желание найти более точное значение числа $\Pi$. Однако, математики средневековой Индии продолжили открывать новые математические методы и теории, которые оказали влияние на развитие математики в мире.

\section-{Нарайана}
Нарайана (также известный как Махараджа Нарайана) был выдающимся математиком и астрономом, который жил в XIV веке на территории современной Индии. Его вклад в развитие математики включает в себя значительные работы по алгебре и теории чисел.

Одно из наиболее известных произведений Нарайаны - "Ганита-каумуди", что можно перевести как "Луна алгебры". Этот текст содержит обширные обсуждения алгебры, включая решение квадратных и кубических уравнений, вычисление сумм арифметических и геометрических прогрессий, а также решение систем уравнений. Он также представил ряд методов для нахождения квадратных и кубических корней.

Нарайана также известен своими работами в области теории чисел. Он исследовал различные свойства чисел, такие как делители, совершенные числа и числа Фибоначчи.

Его вклад в математику был значительным, и его работы продолжают быть источником изучения и вдохновения для математиков в Индии и за ее пределами.

\section-{Рамануджан}
Сриниваса Рамануджан был одним из наиболее выдающихся математиков в истории. Он родился в Индии в 1887 году и проявил удивительные математические способности уже в раннем детстве. Несмотря на отсутствие формального образования в математике, он сделал множество значимых открытий, которые оказали огромное влияние на различные области математики.

Сриниваса Рамануджан внес огромный вклад в различные области математики, включая аналитическую теорию чисел, теорию модулярных форм, теорию рядов и многие другие. Вот некоторые из его значимых достижений:

Теория модулярных форм: Рамануджан сформулировал тысячи новых теорем в области модулярных форм. Его работы в этой области стали основой для многих последующих исследований и имели огромное значение для развития современной теории автоморфных форм.

Теория аналитической теории чисел: Одним из наиболее известных результатов Рамануджана в этой области является его работы над разложением чисел на суммы квадратов, кубов и более высоких степеней. Он также исследовал различные свойства чисел, такие как простые числа, делители и теоремы о делении.

Теория рядов: Рамануджан разработал множество новых теорем и методов решения рядов, включая разложение функций в ряды и аппроксимации функций с помощью рядов.

Решение диофантовых уравнений: Он работал над решением диофантовых уравнений и сформулировал несколько новых методов решения подобных задач.

Теория комбинаторики: Рамануджан внес свой вклад и в область комбинаторики, предлагая новые методы и подходы к решению комбинаторных задач.

В целом, его работы оказали огромное влияние на развитие математики, и он продолжает быть объектом изучения и вдохновения для математиков по всему миру.

\section-{Заключение}
Математика древней и средневековой Индии является одним из важнейших источников математического знания и мудрости. Индийские математики внесли огромный вклад в развитие алгебры, геометрии, теории чисел и других математических дисциплин, и их достижения продолжают вдохновлять ученых и математиков по всему миру.

\newpage
\begin{thebibliography}{2}
	\bibitem{bibl1} Володарский А. И. Математика в древней Индии. // Историко математические исследования. — М.: Наука, 2012.
	\bibitem{bibl2} Брахмагупта // Большой Энциклопедический словарь. 2000.
	Володарский А. И. отв. ред. М. М. Рожанская Ариабхата. 2013.
	\bibitem{bibl3} Hooda D. S., Kapur J. N. Āryabhata : Life a. contributions Жизнь и научная деятельность индийского астронома и математика Ариабхаты. 2014.
	Patwardhan K. S., Naimpally S. A., Singh S. L.
	\bibitem{bibl4} Lilavati of Bhaskaracarya. Delhi, 2011.
	Стройк Д. Я. Краткий очерк истории математики. М. «Наука», 2006.

\end{thebibliography}


\end{document} 