
\documentclass[12pt, a4paper]{article}

\usepackage[utf8]{inputenc}
\usepackage[T1]{fontenc}
\usepackage[russian]{babel}
\usepackage[oglav,spisok,boldsect,eqwhole,figwhole,hyperref,hyperprint,remarks,greekit]{./style/fn2kursstyle}
\graphicspath{{./style/}{./figures/}}

\usepackage{multirow}
\usepackage{supertabular}
\usepackage{multicol}
% Параметры титульного листа
\title{Математика древней и средневековой Индии}
\author{В.\,Г.~Пиневич}
\supervisor{О.\,В.~Пугачев}
\group{ФН2-81Б}
\date{2024}

% Переопределение команды \vec, чтобы векторы печатались полужирным курсивом
\renewcommand{\vec}[1]{\text{\mathversion{bold}${#1}$}}%{\bi{#1}}
\newcommand\thh[1]{\text{\mathversion{bold}${#1}$}}
%Переопределение команды нумерации перечней: точки заменяются на скобки
\renewcommand{\labelenumi}{\theenumi)}
\begin{document}

\maketitle

\tableofcontents



\newpage

\section-{Введение}
Математика древней и средневековой Индии представляет собой уникальную систему знаний и методов, которые оказали значительное влияние на развитие математики в мировой истории. Индия считается одним из родин древней математики, где были разработаны многие математические концепции и методы, которые были впоследствии усвоены и применены математиками других культур.

\section-{Позиционная система счисления}
Одной из важнейших отличительных особенностей математики древней Индии является использование позиционной системы счисления. Позиционная система счисления --- система счисления, в которой значение каждого числового знака (цифры) в записи числа зависит от его позиции (разряда) относительно десятичного разделителя.
Каждая позиция в числе имела свое значение, которое зависело от ее положения. Например, число 532 представляло собой 5 сотен, 3 десятки и 2 единицы. Это позволяло легко представлять и работать с большими числами.
Позиционная система счисления в древней Индии также использовала специальные символы для обозначения разрядов, такие как <<шунья>> (нуль), который был введен в VI веке н.э. Этот символ был ключевым в развитии математики и науки в древней Индии, а также в последующем развитии десятичной системы счисления.
Благодаря этому индийцы смогли применять сложные алгоритмы вычислений, что позволило им делать точные расчеты и решать сложные задачи. Позиционная система в древней Индии основывалась на десятичной системе счисления и использовала символы для представления чисел от 0 до 9. Эта система была развита в древней Индии в VIII веке н.э. и стала основой для многих современных систем счисления. 
Первые позиционные изображения были известны индийским астрономам примерно за 500 лет до Бхаскары I работы. Однако эти числа до Бхаскары I записывались не цифрами, а словами или аллегориями и были организованы в стихи. Например, число 1 было дано как луна, поскольку оно существует только один раз; число 2 было представлено крыльями, близнецами или глазами, поскольку они всегда встречаются парами; число 5 было дано 5 чувствами. Подобно нашей нынешней десятичной системе , эти слова были выровнены таким образом, что каждое число присваивает коэффициент степени десяти, соответствующий его положению, только в обратном порядке: более высокие степени были справа от нижних.
Древнейшая известная запись позиционной десятичной системы обнаружена в Индии в 595 году. В Европе о ней рассказал среднеазиатский математик аль-Хорезми, а поскольку его труд был написан на арабском, то за индийской нумерацией в Европе закрепилось иное название~--~<<арабская>>. 

\section-{Брахмагупта}
Одним из наиболее известных индийских математиков был Брахмагупта, который жил в VII веке. 

В своей работе Брахма-спхута-сиддханта Брахмагупта дал определение нуля как результат вычитания из числа самого числа. Он одним из первых установил правила арифметических операций над положительными и отрицательными числами и нулём, рассматривая при этом положительные числа как имущество, а отрицательные числа как долг. Далее Брахмагупта пытался расширить арифметику дав определение деления на ноль. Согласно Брахмагупте, деление нуля на ноль есть ноль,
деление положительного или отрицательного числа на ноль есть дробь с нулём в знаменателе,
деление нуля на положительное или отрицательное число есть ноль.
Брахмагупта предложил три метода умножения многозначных чисел в столбик (основной и два упрощённых), которые близки к тем, что используются в настоящее время. Основной метод Брахмагупта назвал <<гомутрика>>, что в переводе Ифра означает <<как траектория мочи коровы>>.

Брахмагупта также предложил метод приближённого вычисления квадратного корня, эквивалентный итерационной формуле Ньютона, метод решения некоторых неопределённых квадратных уравнений вида $a x^2 + c = y^2$, метод решения неопределённых линейных уравнений вида $a x + c = b y$, используя метод последовательных дробей.

Он определил сумму квадратов и кубов первых n чисел через сумму первых n чисел, утверждая что <<Сумма квадратов есть сумма чисел, умноженная на удвоенное число шагов, увеличенное на единицу, и делённая на три. Сумма кубов есть квадрат суммы чисел до одного и того же числа>>.

Труды Брахмагупта стали основой для многих последующих открытий в математике.

\section-{Бхаскара II}
Еще одним выдающимся математиком Индии был Бхаскара II, живший в XII веке. Он изучал алгебру, тригонометрию и астрономию, и его труды оказали огромное влияние на развитие математики в мире. Бхаскара II разработал формулы для вычисления площадей и объемов различных геометрических фигур, а также ввел понятие нуля как цифры.

Некоторые из вкладов Бхаскары в математику включают следующее:
\begin{enumerate}
\item Доказательство теоремы Пифагора путем вычисления той же площади двумя разными способами с последующим сокращением членов чтобы получить $a + b = c$.
\item В Лилавати решения квадратичного, кубического и четвертого неопределенного равно поясняются пункты.
Решения неопределенных квадратных уравнений (типа $a x + b = y$).
\item Целочисленные решения линейных и квадратных неопределенных уравнений. Правила, которые он дает, те же самые, что были даны Ренессансом европейскими математиками 17 века
\item Циклическим методом Чакравалы для решения неопределенных уравнений форма $a x + b x + c = y$. Решение этого уравнения традиционно приписывалось Уильяму Браункеру в 1657 году, хотя его метод был сложнее, чем метод чакравалы.
\item Первый общий метод поиска решений задачи $x - ny = 1$ было дано Бхаскарой II.
\item Решения диофантовых уравнений второго порядка, например $61 x + 1 = y$. Это уравнение было поставлено в качестве проблемы в 1657 г. французским математиком Пьером де Ферма, но его решение не было известно в Европе до времен Эйлера в 18 век.
\item Решил квадратные уравнения с более чем одним неизвестным и нашел отрицательные и иррациональные решения.
\item Предварительная концепция бесконечно малого исчисления, наряду с заметным вкладом в интегральное исчисление.
\item Задуманное дифференциальное исчисление, после обнаружения приближения производная и дифференциальный коэффициент.
\item Заявленная теорема Ролля, частный случай одной из важнейших теорем анализа, Теорема о среднем значении. Следы общей теоремы о среднем значении также найдены в его работах.
\item Вычислял производные тригонометрических функций и формул.
\item В Сиддханта-Широмани Бхаскара разработал сферическую тригонометрию вместе с рядом других тригонометрических результатов.
\item Оценка числа $\Pi$.
\end{enumerate}

\section-{Средневековье}
В средневековом периоде математика в Индии продолжала развиваться, хотя в это время страна потеряла свою ведущую роль в математической науке. XVI век был отмечен крупными открытиями в теории разложения в ряды, переоткрытыми в Европе 100—200 лет спустя. В том числе — ряды для синуса, косинуса и арксинуса. Поводом к их открытию послужило, видимо, желание найти более точное значение числа $\Pi$. Однако, математики средневековой Индии продолжили открывать новые математические методы и теории, которые оказали влияние на развитие математики в мире.

\section-{Заключение}
Математика древней и средневековой Индии является одним из важнейших источников математического знания и мудрости. Индийские математики внесли огромный вклад в развитие алгебры, геометрии, теории чисел и других математических дисциплин, и их достижения продолжают вдохновлять ученых и математиков по всему миру.

\end{document} 