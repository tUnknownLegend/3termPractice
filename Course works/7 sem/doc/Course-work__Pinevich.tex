
\documentclass[12pt, a4paper]{article}

\usepackage[utf8]{inputenc}
\usepackage[T1]{fontenc}
\usepackage[russian]{babel}
\usepackage[oglav,spisok,boldsect,eqwhole,figwhole,hyperref,hyperprint,remarks,greekit]{./style/fn2kursstyle}
\graphicspath{{./style/}{./figures/}}

\usepackage{multirow}
\usepackage{supertabular}
\usepackage{multicol}
\usepackage{amsmath}
\usepackage{afterpage}
% Параметры титульного листа
\title{Решение дифференциального уравнения\\ Рейнольдса методом конечных элементов}
\author{В.\,Г.~Пиневич}
\supervisor{А.\,В.~Селиванов}
\group{ФН2-71Б}
\date{2024}

% Переопределение команды \vec, чтобы векторы печатались полужирным курсивом
\renewcommand{\vec}[1]{\text{\mathversion{bold}${#1}$}}%{\bi{#1}}
\newcommand\thh[1]{\text{\mathversion{bold}${#1}$}}
%Переопределение команды нумерации перечней: точки заменяются на скобки
\renewcommand{\labelenumi}{\theenumi)}
\begin{document}

\maketitle

\tableofcontents



\newpage

\section-{Введение}


\section{Постановка задачи}
Задача данной работы --- вывести, а затем найти решение дифференциального уравнения Рейнольдса методом конечных элементов.
\begin{equation}
	\label{reinolts-task}
\frac{\partial}{\partial x} \left(h^3 \frac{\partial p}{\partial x} \right) + \frac{\partial}{\partial z} \left(h^3 \frac{\partial p}{\partial z} \right) = 6 \mu U \frac{\partial h}{\partial x} \text{, }
\end{equation}
где $h = h(x)$ --- толщина слоя, $p = p(x, z)$ --- давление, $\mu$ --- коэффициент вязкости. Граничные условия: $U$ --- скорость в направлении $x$ на одной из пластин, $p_b$ --- повышенное давление, $p_l$ --- пониженное давление.

\section{Вывод уравнения Рейнольдса}

Гидродинамические уравнения несжимаемой жидкости с
внутренним трением могут быть представлены в очень простой
форме, если пренебречь силами, пропорциональными массам,
равно как и силами инерции.

Обозначая через $х, у, z$ прямоугольные координаты точки, через $р$ -- гидродинамическое давление в этой точке,

\[
\begin{cases}
	p_{xy}, p_{xz}; \\
	p_{yx}, p_{yz}; \\
	p_{zx}, p_{zy}.
\end{cases}
\]

силы трения, перпендикулярные к оси, обозначенной первой буквой индекса и параллельные оси, обозначенной второй буквой индекса $u, \nu, \omega$ -- проекции скорости на осях $x, y, z \text{. }$$\mu$ --- коэффициент внутреннего трения жидкости, можно написать три группы следующих уравнений:
\begin{enumerate}
	\item Группа, определяющая гидродинамическое давление в
	точке $x, y, z$:
	\begin{equation}
		\label{eqfi}
		\begin{cases}
			\frac{\partial p}{\partial x} = \mu \left( \frac{\partial^2 u}{\partial^2 x} + \frac{\partial^2 u}{\partial^2 y} + \frac{\partial^2 u}{\partial^2 z} \right), \\
				\frac{\partial p}{\partial y} = \mu \left( \frac{\partial^2 \nu}{\partial^2 x} + \frac{\partial^2 \nu}{\partial^2 y} + \frac{\partial^2 \nu}{\partial^2 z} \right), \\
					\frac{\partial p}{\partial z} = \mu \left( \frac{\partial^2 \omega}{\partial^2 x} + \frac{\partial^2 \omega}{\partial^2 y} + \frac{\partial^2 \omega}{\partial^2 z} \right).
		\end{cases}
	\end{equation}
\item Группа, определяющая силы трения в той же точке:
\begin{equation}
	\label{eqsi}
	\begin{cases}
		p_{yz} = p_{zy} = \mu \left(\frac{\partial \omega}{\partial y} + \frac{\partial \nu}{\partial z} \right), \\
			p_{zx} = p_{xz} = \mu \left( \frac{\partial \omega}{\partial x} +  \frac{\partial u}{\partial z} \right), \\
				p_{xy} = p_{yx} = \mu \left(  \frac{\partial u}{\partial y} + \frac{\partial \nu}{\partial x} \right).
	\end{cases}
\end{equation}
\item Условие несжимаемости жидкости, выраженное урав
нением: 
\begin{equation}
	\label{eqthi}
	\frac{\partial u}{\partial x} + \frac{\partial \nu}{\partial y} + \frac{\partial \omega}{\partial z} = 0.
	\end{equation}

\end{enumerate}

Примем, что скорость $\nu = 0$, поскольку она мала по сравнению со скоростями $u = 0$, $\omega = 0$.

Изменения скоростей и и со при заданном значении $y$ для всех изменений $ x $ и $z$ могут рассматриваться как чрезмерно малые, поэтому причем
\[
\frac{\partial^2 u}{\partial^2 x} = 0, 
\frac{\partial^2 u}{\partial^2 z} = 0, 
\frac{\partial^2 \omega}{\partial^2 x} = 0, 
\frac{\partial^2 \omega}{\partial^2 z} = 0. 
\]

Ограничиваясь приближенным решением, которое можно
получить при указанных выше предположениях, уравнения \eqref{eqfi}, \eqref{eqsi} и \eqref{eqthi} могут быть приведены к следующей форме.
\begin{equation}
	\label{secondinitialeq}
	\begin{cases}
		\frac{\partial p }{\partial x} = \mu \frac{\partial^2 u}{\partial^2 y}, \\
		\frac{\partial p }{\partial y} = 0, \\
		\frac{\partial p }{\partial z} = \mu \frac{\partial^2 \omega}{\partial^2 y}.
	\end{cases}
\end{equation}
\begin{equation}
	\label{secinitialeq}
	\begin{cases}
		p_{yz} = p_{xy} = \mu \frac{\partial \omega}{\partial y}, \\
		p_{zx} = p_{xz} = 0, \\
		p_{xy} = p_{yx} = \mu \frac{\partial u}{\partial y}.
	\end{cases}
\end{equation}
\begin{equation*}
	\frac{\partial u}{\partial x} + \frac{\partial \nu}{\partial y} + \frac{\partial \omega}{\partial z} = 0.
\end{equation*}

Для определения давления необходимо интегрировать выражения \eqref{secondinitialeq}, \eqref{secinitialeq}. Для этого определим граничные условия.
Для $y = 0$ имеем
\[
u = U_0, \nu = 0, \omega = 0.
\]
Для $y = h$ имеем
\[
u = U_1, \nu = U_1 - U_1 \frac{\partial  h}{\partial h}, \omega = 0.
\]
На некотором контуре $f(x, y) = 0$ имеем
\[
p = p_0.
\]

Поскольку $p$ не зависит от $y$, то интегрирование уравнений \eqref{secondinitialeq} приводит к уравнениям
\begin{equation}
	\label{1-inte-inti}
	\begin{cases}
		u = \frac{1}{2 \mu} \frac{\partial p}{\partial x} \left( y - h \right) y + U_0 \frac{h - y}{h} + U_1 \frac{y}{h},\\
		\omega = \frac{1}{2 \mu} \frac{\partial p}{\partial z} (y - h) y.
	\end{cases}
\end{equation} 
Первые производные вторых членов этих уравнений, перене
сенные в соответствующие уравнения группы \eqref{secinitialeq}, приводят
к уравнениям
\begin{equation}
	\label{sec-init-eq}
	\begin{cases}
		p_{yz} = p_{zy} = \frac{1}{2} \frac{\partial p}{\partial z} \left( 2y - h \right), \\
		p_{xy} = p_{yz} = \frac{1}{2} \frac{\partial p}{\partial x} \left( 2y - h \right) + \mu \frac{U_1 - U_0}{h}.
	\end{cases}
\end{equation}

Если $р$ считать независимым от $z$, то четыре последних
уравнения сокращаются до двух: первое из группы~\eqref{1-inte-inti} и
второе из группы~\eqref{sec-init-eq}.

Взяв производные от первого из этих уравнений по $x$ и
от второго по $z$ и подставляя это в уравнение~\eqref{eqthi}, находим, что
\begin{equation*}
		\frac{\partial \nu}{\partial y} = - \frac{1}{2 \mu} \left( \frac{\partial}{\partial x} \left( \frac{\partial p}{\partial x} (y - x) y \right) + \frac{\partial}{\partial z} \left( \frac{\partial p}{\partial z} (y - h) h \right) - \frac{\partial}{\partial x} \left( U_0 \frac{h - y}{h} + U_1 \frac{y}{h} \right) \right).
\end{equation*}

Интегрируя это уравнение в пределах от г/ = 0 до г/ = Л и
принимая во внимание условия [6], получаем
\begin{equation*}
	\frac{\partial}{\partial x} \left( h^3 \frac{\partial p}{\partial x} \right) + \frac{\partial}{\partial z} \left( h^3 \frac{\partial p}{\partial z} \right) + \frac{\partial}{\partial x} \left( h^3 \frac{p}{x} \right) = 6 \mu \left( (U_0 - U_1) \frac{\partial h}{\partial x} \right) + 2 V_1.
\end{equation*}
$2 V_1$ используется для учёта движений одной из стенок зазора, меняющих значение функции. Если пренебречь этим, и обозначить $U_0 - U_1$ как $U$, то получим искомое уравнение~\eqref{reinolts-task}.

\newpage
\section-{Заключение}

\begin{enumerate}	
	\item .
\end{enumerate}
 
\newpage
\begin{thebibliography}{2}
\bibitem{metoda_bmstu} Петров Н. Гидродинамическая теория смазки, М.: Из-во академии наук СССР, 1948. --- 558~с.

\end{thebibliography}

\end{document} 