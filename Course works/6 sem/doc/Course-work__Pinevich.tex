
\documentclass[12pt, a4paper]{article}

\usepackage[utf8]{inputenc}
\usepackage[T1]{fontenc}
\usepackage[russian]{babel}
\usepackage[oglav,spisok,boldsect,eqwhole,figwhole,hyperref,hyperprint,remarks,greekit]{./style/fn2kursstyle}
\graphicspath{{./style/}{./figures/}}

\usepackage{multirow}
\usepackage{supertabular}
\usepackage{multicol}
\usepackage{amsmath}
% Параметры титульного листа
\title{Решение жесткой системы\\ дифференциальных уравнений}
\author{В.\,Г.~Пиневич}
\supervisor{А.\,В.~Котович}
\group{ФН2-61Б}
\date{2023}

% Переопределение команды \vec, чтобы векторы печатались полужирным курсивом
\renewcommand{\vec}[1]{\text{\mathversion{bold}${#1}$}}%{\bi{#1}}
\newcommand\thh[1]{\text{\mathversion{bold}${#1}$}}
%Переопределение команды нумерации перечней: точки заменяются на скобки
\renewcommand{\labelenumi}{\theenumi)}
\begin{document}

\maketitle

\tableofcontents



\newpage

\section-{Введение}
Проблема решения задачи жестких систем дифференциальных уравнений возникает во многих сферах науки и техники. Существует большое количество различных методов решения таких задач. В данной работе будет рассмотрено решение задачи методом <метод>.

\section{Постановка задачи}
Задача данной работы --- найти решение модели химических реакций Робертсона. 
\begin{equation}
	\label{taskDef}
	\begin{cases}
		y_1 = -0,04 y_1 + 10^4 y_2 y_3,\\
		y_2 = 0,04 y_1 - 10^4 y_2 y_3 - 3 * 10^7 y_2^2,\\
		y_3 = 3 * 10^7 y_2^2.
	\end{cases}
\end{equation}
Кроме того, требуется построить фазовые траектории для данной задачи.

\subsection{Жесткая система}
Пусть есть система дифференциальных уравнений 
\begin{equation}
	\label{example_eq}
	y_t = f(t, y), 0 <= t <= T, y(0) = y_0.
\end{equation}

Система называется жесткой, если для всех t, y (т. е. на решениях~(\ref{example_eq})), собственные значения матрицы A удовлетворяют условиям~\cite{metoda_bmstu}.

\begin{equation}
	\begin{cases}
	\frac{max | Re \lambda_j |}
	{min | Re \lambda_j |} >> 1, Re \lambda_j < 0,\\
	max | Im \lambda_j | << max | Re \lambda_j |, j, k = 1, ..., J.
	\end{cases}
\end{equation}

Схема называется абсолютно устойчивой, если $|q(\sigma)| <= 1 $ выполняется при всех значениях.

Схема называется A-устойчивой, если кривая $|q(\sigma)| = 1$ лежит в правой полуплоскости $\sigma$.

\section-{Метод}
\subsection{Описание метода}

\subsection{Аппроксимация}

\subsection{Устойчивость}



\newpage
\section-{Заключение}
 
\newpage
\begin{thebibliography}{2}
\bibitem{metoda_bmstu} metoda

\end{thebibliography}

\end{document} 