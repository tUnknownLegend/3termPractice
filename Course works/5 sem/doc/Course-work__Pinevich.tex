
\documentclass[12pt, a4paper]{article}

\usepackage[utf8]{inputenc}
\usepackage[T1]{fontenc}
\usepackage[russian]{babel}
\usepackage[oglav,spisok,boldsect,eqwhole,figwhole,hyperref,hyperprint,remarks,greekit]{./style/fn2kursstyle}
\graphicspath{{./style/}{./figures/}}

\usepackage{multirow}
\usepackage{supertabular}
\usepackage{multicol}
% Параметры титульного листа
\title{Кручение стержня\\ прямоугольного сечения}
\author{В.\,Г.~Пиневич}
\supervisor{А.\,В.~Котович}
\group{ФН2-51Б}
\date{2023}

% Переопределение команды \vec, чтобы векторы печатались полужирным курсивом
\renewcommand{\vec}[1]{\text{\mathversion{bold}${#1}$}}%{\bi{#1}}
\newcommand\thh[1]{\text{\mathversion{bold}${#1}$}}
%Переопределение команды нумерации перечней: точки заменяются на скобки
\renewcommand{\labelenumi}{\theenumi)}
\begin{document}

\maketitle

\tableofcontents



\newpage

\section-{Введение}
Проблема вычисления уравнение прогиба балки возникает во многих задачах, в частности в строительной механике. В силу наличия большого числа действующих сил и моментов решение такой задачи классическим методом, т.е. вычислением дифференциального уравнения, вызывает сложности в виду большого числа граничных условий. Однако с развитием теории обобщенных функций и сопротивления материалов был найден более удобный способ расчета прогиба балки. Данная работа посвящена изучению методов решения подобных задач, выделение их недостатков и преимуществ.

\section{Постановка задачи}
Для расчета уравнения гибкого стержня необходимо ввести несколько понятий. $q$~---~распределенная нагрузка, $Q$~---~сосредоточенная сила, $M$~---~момент.

\section-{Заключение}
В ходе выполнения курсовой были изучены методы интегрирования и обобщенных функций нахождения уравнения упругого изгиба стержня.	
С помощью этих методов были решены два типа задач, их результаты оказались идентичны.
Метод интегрирования является более трудоемким и менее удобным по сравнению с методом обобщенных функций, так как требует учета большего количества граничных условий и большего объема вычислений.
\newpage
\begin{thebibliography}{3}
\bibitem{Feofociev} В.И. Феодосьев Сопротивление материалов: учеб. для вузов. --- 10-е изд., перераб. и доп. --- М.: Изд-во МГТУ им. Н.Э.Баумана, 1999. --- 590 с.	
	
\end{thebibliography}

\end{document} 