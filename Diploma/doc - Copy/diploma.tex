
\documentclass[a4paper,14pt]{extarticle}

\usepackage[utf8]{inputenc}
\usepackage[T1]{fontenc}
\usepackage[english,russian]{babel}
\usepackage[oglav,spisok,boldsect,eqwhole,figwhole,hyperref,hyperprint,remarks,greekit]{./style/fn2dipstyle}
\graphicspath{{./style/}{./figures/}}

\usepackage{setspace}
\setstretch{1.5}
\usepackage[left=3cm,right=1cm, top=2cm,bottom=2cm]{geometry}
\usepackage{pdfpages}
\pagestyle{plain}
\usepackage{mathtools}

% Центрированные таблицы фиксированной ширины %
\usepackage{tabularx} % also loads 'array' package
\newcolumntype{C}{>{\centering\arraybackslash}X} % centered version of 'X' columns

\usepackage{relsize}
\newcommand{\smallalpha}{\mathsmaller{(} \alpha \mathsmaller{)}}
\newcommand{\smallbeta}{\mathsmaller{(} \beta \mathsmaller{)}}
\newcommand{\avg}[1]{\left\langle #1 \right\rangle}
\usepackage[titles]{tocloft}

\usepackage{multirow}
\usepackage{supertabular}
\usepackage{multicol}
\usepackage{amsmath}
\usepackage{afterpage}
\usepackage{amsmath}
% Параметры титульного листа
%\title{Решение уравнения Рейнольдса\\ в рамках теории газовой смазки \\ методом конечных элементов}
%\author{В.\,Г.~Пиневич}
%\supervisor{А.\,В.~Селиванов}
%\group{ФН2-81Б}
%\date{2024}

% Переопределение команды \vec, чтобы векторы печатались полужирным курсивом
\renewcommand{\vec}[1]{\text{\mathversion{bold}${#1}$}}%{\bi{#1}}
\newcommand\thh[1]{\text{\mathversion{bold}${#1}$}}
%Переопределение команды нумерации перечней: точки заменяются на скобки
\renewcommand{\labelenumi}{\theenumi)}
\begin{document}
	
\includepdf[pages=1]{title.pdf}
	
	
\section-{Реферат}	

Расчетно-пояснительная записка 30 с., 16 рис., 4 табл., 5 источников, 1
прил.

МЕТОД КОНЕЧНЫХ ЭЛЕМЕНТОВ, УРАВНЕНИЕ РЕЙНОЛЬДСА,
АЭРОУПРУГАЯ МОДЕЛЬ.

Составлено решение уравнение Рейнольдса методом конечных элементов. Решение может быть получено как для переменной, так и для постоянной правой части. 

Численные результаты, полученные с помощью разработанной программы были проверены с помощью постановки правой части, а также сравнены со результатами вычисления Wolfram Mathematica

Получены решения задачи Рейнольдса для различных наклонов зазора и граничных условий.

Модель решения уравнения Рейнольдса была дополнена с помощью ограничения движения пластины зазора пружиной. Найдено положение равновесия, проверена устойчивость этого положения.

\newpage

\tableofcontents



\newpage

\section{Введение}
Задачи расчёта подшипников газодинамического типа с зазорами разнообразной формы до сих пор остаются весьма актуальными. По сути, они сводятся к изучению газового смазочного слоя в тонком зазоре произвольной формы. Решением таких задач занимается гидродинамическая теория смазки. Этот раздел механики жидкости и газа начал развиваться в конце XIX века вслед за потребностями техники. Начало теоретическому исследованию течений в тонких зазорах положили работы Н. П. Петрова и британского учёного Осборна Рейнольдса, уточнённые и доведённые до возможности практического применения А. Зоммерфельдом, А. Мичелем. Дальнейшие исследования позволили распространить результаты созданной Рейнольдсом теории на газодинамические подшипники. Были даже предприняты успешные попытки получения общего вида уравнения Рейнольдса для смазочного слоя без привязки к конкретной системе координат. 
Помимо этого для создания газотурбинного двигателя в настоящее время требуется уплотнение зазоров во многих элементах конструкции. Это необходимо для улучшения характеристик двигателя и его надежности. Уплотнители разделяют на контактные и бесконтактные. Они имеют понятные преимущества и недостатки по сравнению друг с другом. По скольку трение много выше при использовании контактного уплотнителя, это приводит к его более быстрому износу, однако позволяет создать постоянный контакт с уплотняемой деталью и получить лучшую эффективность. С другой стороны, бесконтактный уплотнитель более долговечен, но уровень утечек при их использовании выше. Для расчета уплотнений также используются математическое моделирование.
Данная работа посвящена получению пригодной к использованию в практических задачах формы уравнения Рейнольдса, решению его методом конечных элементов и поиску равновесного состояния для системы дополненной пружиной как ограничителем сверху.


\subsection{Постановка задачи}
Задача данной работы --- вывести, а затем показать методику решение дифференциального уравнения Рейнольдса методом конечных элементов.
\begin{equation}
	\label{reinolts-task}
\frac{\partial}{\partial x} \left(h^3 \frac{\partial p}{\partial x} \right) + \frac{\partial}{\partial z} \left(h^3 \frac{\partial p}{\partial z} \right) = 6 \mu U \frac{\partial h}{\partial x} \text{, }
\end{equation}
где $h = h(x)$ --- толщина слоя, $p = p(x, z)$ --- давление, $\mu$ --- коэффициент вязкости. Граничные условия: $U$ --- скорость в направлении $x$ на одной из пластин, $p_{\text{в}}$ --- повышенное давление, $p_{\text{н}}$ --- пониженное давление. 

Размеры области зададим 5 мм по ширине и 5 мм по длине. Будем использовать коэффициент вязкости воды $\mu = 8.90 \cdot 10^{-4} \text{ Па} \cdot \text{с}$.
\newline Давление возьмем $p_{\text{н}} = 100 \text{ кПа}, p_{\text{в}} = 150 \text{ кПа}$. Скорость будет меняться в зависимости от специфики расчетов. Схематичное изображение области задачи и граничных условий изображено на рис.~\ref{obl_resh}.

\begin{figure}[!htbp]
	\center{\includegraphics[width=\textwidth, height=0.5\textwidth]{taskGU.png}}
	\caption{Схема области решения и граничных условий уравнения Рейнольдса}
	\label{obl_resh}
\end{figure}

\subsection{Получение уравнения Рейнольдса}

Уравнение Рейнольдса является одним из основных уравнений в области механики жидкостей и газов. Оно было названо в честь физика Осгуда Рейнольдса, который впервые сформулировал его в середине XIX века.
Уравнение Рейнольдса имеет широкое применение в различных областях науки и техники, таких как гидродинамика, аэродинамика, теплообмен. Это уравнение является основой для предсказания и анализа различных явлений, связанных с течением жидкостей и газов.
Гидродинамические уравнения несжимаемой жидкости с
внутренним трением могут быть представлены в очень простой
форме, если пренебречь силами, пропорциональными массам,
равно как и силами инерции.

Обозначая через $x$, $y$, $z$ прямоугольные координаты точки, через $p$ -- гидродинамическое давление в этой точке,

Силы трения $p_{xy}, p_{xz}, p_{yx}, p_{yz}, p_{zx}, p_{zy}$, перпендикулярные к оси силы, обозначенной первой буквой индекса и параллельные оси силы, обозначенной второй буквой индекса.

Обозначим проекции скорости на осях $x, y, z \text{ соответственно}$ $u, \nu, \omega$.

Введем $\mu$ как коэффициент внутреннего трения жидкости.ы
 Тогда можно записать три системы уравнений:
\begin{enumerate}
	\item Система, определяющая гидродинамическое давление в
	точке $x, y, z$:
	\begin{equation}
		\label{eqfi}
		\begin{cases}
			\frac{\partial p}{\partial x} = \mu \left( \frac{\partial^2 u}{\partial x^2} + \frac{\partial^2 u}{\partial y^2} + \frac{\partial^2 u}{\partial z^2} \right), \\
				\frac{\partial p}{\partial y} = \mu \left( \frac{\partial^2 \nu}{\partial x^2} + \frac{\partial^2 \nu}{\partial y^2} + \frac{\partial^2 \nu}{\partial z^2} \right), \\
					\frac{\partial p}{\partial z} = \mu \left( \frac{\partial^2 \omega}{\partial x^2} + \frac{\partial^2 \omega}{\partial y^2} + \frac{\partial^2 \omega}{\partial z^2} \right).
		\end{cases}
	\end{equation}
\item Система, определяющая силы трения в той же точке:
\begin{equation}
	\label{eqsi}
	\begin{cases}
		p_{yz} = p_{zy} = \mu \left(\frac{\partial \omega}{\partial y} + \frac{\partial \nu}{\partial z} \right), \\
			p_{zx} = p_{xz} = \mu \left( \frac{\partial \omega}{\partial x} +  \frac{\partial u}{\partial z} \right), \\
				p_{xy} = p_{yx} = \mu \left(  \frac{\partial u}{\partial y} + \frac{\partial \nu}{\partial x} \right).
	\end{cases}
\end{equation}
\item Условие несжимаемости жидкости, выраженное уравнением: 
\begin{equation}
	\label{eqthi}
	\frac{\partial u}{\partial x} + \frac{\partial \nu}{\partial y} + \frac{\partial \omega}{\partial z} = 0.
	\end{equation}

\end{enumerate}

Примем, что скорость $\nu = 0$, поскольку она мала по сравнению со скоростями $u = 0$, $\omega = 0$.

Изменения скоростей $u$ и $\omega$ со при заданном значении $y$ для всех изменений $ x $ и $z$ могут рассматриваться как чрезмерно малые, поэтому примем
\[
\frac{\partial^2 u}{\partial x^2} = 0, 
\frac{\partial^2 u}{\partial z^2} = 0, 
\frac{\partial^2 \omega}{\partial x^2} = 0, 
\frac{\partial^2 \omega}{\partial z^2} = 0. 
\]

Ограничиваясь приближенным решением, которое можно
получить при указанных выше предположениях, уравнения \eqref{eqfi}, \eqref{eqsi} и \eqref{eqthi} могут быть приведены к следующей форме.
\begin{equation}
	\label{secondinitialeq}
	\begin{cases}
		\frac{\partial p }{\partial x} = \mu \frac{\partial^2 u}{\partial y^2}, \\
		\frac{\partial p }{\partial y} = 0, \\
		\frac{\partial p }{\partial z} = \mu \frac{\partial^2 \omega}{\partial y^2}.
	\end{cases}
\end{equation}
\begin{equation}
	\label{secinitialeq}
	\begin{cases}
		p_{yz} = p_{xy} = \mu \frac{\partial \omega}{\partial y}, \\
		p_{zx} = p_{xz} = 0, \\
		p_{xy} = p_{yx} = \mu \frac{\partial u}{\partial y}.
	\end{cases}
\end{equation}
\begin{equation*}
	\frac{\partial u}{\partial x} + \frac{\partial \nu}{\partial y} + \frac{\partial \omega}{\partial z} = 0.
\end{equation*}

Для определения давления необходимо интегрировать выражения \eqref{secondinitialeq}, \eqref{secinitialeq}. Для этого определим граничные условия.

\noindent Для $y = 0$ имеем
\[
\begin{cases}
u = U_0, \\
 \nu = 0, \\
  \omega = 0.
\end{cases}
\]

\noindent Для $y = h$ имеем
\[
\begin{cases}
u = U_1, \\ 
\nu = U_1 - U_1 \frac{\partial  h}{\partial h}, \\ \omega = 0.
\end{cases}
\]

Поскольку $p$ не зависит от $y$, то интегрирование уравнений \eqref{secondinitialeq} приводит к уравнениям
\begin{equation}
	\label{1-inte-inti}
	\begin{cases}
		u = \frac{1}{2 \mu} \frac{\partial p}{\partial x} \left( y - h \right) y + U_0 \frac{h - y}{h} + U_1 \frac{y}{h},\\
		\omega = \frac{1}{2 \mu} \frac{\partial p}{\partial z} (y - h) y.
	\end{cases}
\end{equation} 
Первые производные вторых членов этих уравнений, перенесенные в соответствующие уравнения группы \eqref{secinitialeq}, приводят
к уравнениям
\begin{equation}
	\label{sec-init-eq}
	\begin{cases}
		p_{yz} = p_{zy} = \frac{1}{2} \frac{\partial p}{\partial z} \left( 2y - h \right), \\
		p_{xy} = p_{yz} = \frac{1}{2} \frac{\partial p}{\partial x} \left( 2y - h \right) + \mu \frac{U_1 - U_0}{h}.
	\end{cases}
\end{equation}

Если давление $p$ считать независимым от координаты $z$, то четыре последних
уравнения сокращаются до двух: первое из системы~\eqref{1-inte-inti} и
второе из системы~\eqref{sec-init-eq}.

Взяв производные от первого из этих уравнений по $x$ и
от второго по $z$ и подставляя это в уравнение~\eqref{eqthi}, находим, что
\begin{equation*}
		\frac{\partial \nu}{\partial y} = - \frac{1}{2 \mu} \left( \frac{\partial}{\partial x} \left( \frac{\partial p}{\partial x} (y - x) y \right) + \frac{\partial}{\partial z} \left( \frac{\partial p}{\partial z} (y - h) h \right) - \frac{\partial}{\partial x} \left( U_0 \frac{h - y}{h} + U_1 \frac{y}{h} \right) \right).
\end{equation*}

Интегрируя это уравнение в пределах от $y = 0$ до $y = h$ и
принимая во внимание граничные условия, получаем
\begin{equation*}
	\frac{\partial}{\partial x} \left( h^3 \frac{\partial p}{\partial x} \right) + \frac{\partial}{\partial z} \left( h^3 \frac{\partial p}{\partial z} \right) = 6 \mu \left( (U_0 - U_1) \frac{\partial h}{\partial x} \right) + 2 V_1.
\end{equation*}
$2 V_1$ используется для учёта движений одной из стенок зазора, меняющих значение функции. Если пренебречь этим, и обозначить $U_0 - U_1$ как $U$, то получим искомое уравнение~\eqref{reinolts-task}.

\section{Решения уравнения Рейнольдса}

Решение уравнения Рейнольдса~\eqref{reinolts-task} будем искать с помощью метода конченых элементов.
Изначально этот метод возник из необходимости решать задачи строительной механики. 
Существенный толчок в своём развитии МКЭ получил в 1963 году после того, как было доказано, что его можно рассматривать как один из вариантов распространённого в строительной механике метода Рэлея-Ритца, который путём минимизации потенциальной энергии сводит задачу к системе линейных уравнений равновесия. Область применения МКЭ значительно расширилась, когда было установлено, что уравнения, определяющие элементы в задачах, могут быть легко получены с помощью вариантов метода взвешенных невязок, таких как метод Галеркина или метод наименьших квадратов. Это сыграло важную роль в теоретическом обосновании МКЭ, так как позволило применять его при решении многих типов дифференциальных уравнений. Таким образом, метод конечных элементов превратился в общий метод численного решения дифференциальных уравнений или систем дифференциальных уравнений.
Основная идея метода конечных элементов состоит в том, что любую непрерывную величину, такую как температура, давление и перемещение, можно аппроксимировать дискретной моделью, которая строится на множестве кусочно-непрерывных функций, определенных на конечном числе подобластей. Кусочно-непрерывные функции определяются с помощью значений непрерывной величины в конечном числе точек рассматриваемой области.
В общем случае непрерывная величина заранее неизвестна и нужно определить значения этой величины в некоторых внутренних точках области. Дискретную модель, однако, не сложно построить, если сначала предположить, что числовые значения этой величины в каждой внутренней точке области известны. После этого можно перейти к общему случаю. Итак, при построении дискретной модели непрерывной величины поступают следующим образом:
\begin{enumerate}
	\item В рассматриваемой области фиксируется конечное число точек. Эти точки называются узловыми точками или просто узлами.
	\item Значение непрерывной величины в каждой узловой точке считается переменной, которая должна быть определена.
	\item Область определения непрерывной величины разбивается на конечное число подобластей, называемых элементами. Эти элементы имеют общие узловые точки и в совокупности аппроксимируют форму области.
	\item Выбор функций формы и аппрокисмирующей функции.
	\item Построение локальной матрицы.
	\item Построение глобальной матрицы путем сшивания локальных матриц элементов друг с другом.
	\item Учет граничных условий путем заменой коэффициентов в полученной матрице или же вектора правых частей.
	\item Решение системы линейных уравнений для получения решения в узловых точках.
\end{enumerate}

\subsection{Слабая форма Галеркина}

Существуют разные подходы к реализации идеи метода конечных элементов. В данной работе рассмотрим методику решения с помощью слабой формы Галеркина. Для разбиения области на элементы требуется выбрать форму элемента. Поскольку рассматриваемая область прямоугольная, то удобно взять элементы прямоугольной формы. 

\begin{figure}[!htbp]
	\center{\includegraphics[width=0.25\textwidth, height=0.25\textwidth]{base_element.png}}
	\caption{Линейный прямоугольный конечный элемент}
	\label{base_element}
\end{figure}
\begin{figure}[!htbp]
	\center{\includegraphics[width=0.25\textwidth, height=0.25\textwidth]{base_element_proection.png}}
	\caption{Проекция расчётной области}
	\label{base_element_proection}
\end{figure}

Пусть $l$ --- горизонтальная длинная области, а $h$ --- вертикальная. Тогда функции формы будут иметь следующий вид:
\begin{equation}
	\begin{cases}
		N_1 = 1 - \frac{x}{l} - \frac{z}{h} + \frac{x  z}{l  h}, \\
		N_2 = \frac{x}{l} - \frac{x  z}{l  h}, \\
		N_3 = \frac{x  z}{l h}, \\
		N_4 = \frac{z}{h} - \frac{x  z}{l  h}. \\
	\end{cases}
\label{form-func}
\end{equation}

\noindent
Аппроксимирующую функцию зададим в виде:
\begin{equation*}
	\phi = c_0 N_1 + c_1 N_2 + c_2 N_3 + c_3 N_4.
\end{equation*}
\noindent
Далее будем искать локальную матрицу из выражения:
\begin{equation}
	\label{init_eq}
	\int_{S_i} {[N]^T \left(\frac{\partial}{\partial x} \left(h^3 \frac{\partial p}{\partial x} \right) + \frac{\partial}{\partial z} \left(h^3 \frac{\partial p}{\partial z} \right) - 6 \mu U \frac{\partial h}{\partial x}\right) dx dz} = 0.
\end{equation}
$S_i$ --- область, содержащая элемент. 
Подставим аппрокисимирующую функцию $\phi$ вместо $p$.
\begin{equation*}
	\int_{S_i} {[N]^T \left(\frac{\partial}{\partial x} \left(h^3 \frac{\partial \phi}{\partial x} \right) + \frac{\partial}{\partial z} \left(h^3 \frac{\partial \phi}{\partial z} \right) - 6 \mu U \frac{\partial h}{\partial x}\right) dx dz} = 0.
\end{equation*}
Понизим порядок производной с помощью функции дифференцирования произведения:
\begin{equation*}
	 \frac{\partial}{\partial x} \left( [N]^T h^3 \frac{\partial \phi}{\partial x} \right)  = [N]^T \frac{\partial}{\partial x} \left(h^3 \frac{\partial \phi}{\partial x} \right) + \frac{\partial[N]^T}{\partial x} \left(h^3 \frac{\partial \phi}{\partial x} \right),
\end{equation*}
\begin{equation*}
	 \frac{\partial}{\partial z} \left( [N]^T h^3 \frac{\partial \phi}{\partial z} \right)  = [N]^T \frac{\partial}{\partial z} \left(h^3 \frac{\partial \phi}{\partial z} \right) + \frac{\partial[N]^T}{\partial z} \left(h^3 \frac{\partial \phi}{\partial z} \right).
\end{equation*}

Подставим полученные выражения в исходный интеграл~\eqref{init_eq}:
\begin{equation*}
\int_{S_i} {\left(\frac{\partial}{\partial x} \left( [N]^T h^3 \frac{\partial \phi}{\partial x} \right) - \frac{\partial[N]^T}{\partial x} \left(h^3 \frac{\partial \phi}{\partial x} \right) \right) dxdz} + 
\end{equation*}
\begin{equation*}
 + \int_{S_i} {\left(\frac{\partial}{\partial z} \left( [N]^T h^3 \frac{\partial \phi}{\partial z} \right) - \frac{\partial[N]^T}{\partial z} \left(h^3 \frac{\partial \phi}{\partial z} \right) - [N]^T 6 \mu U \frac{\partial h}{\partial x}\right) dxdz} = 0.
\end{equation*}

С помощью формулы Грина можно заменить третье и первое слагаемое под знаком интеграла
\begin{equation*}
	\int_{S_i} {\left(\frac{\partial}{\partial x} \left( [N]^T h^3 \frac{\partial \phi}{\partial x} \right) + \frac{\partial}{\partial z} \left( [N]^T h^3 \frac{\partial \phi}{\partial z} \right)  \right)   dxdz} =
\end{equation*}
\begin{equation*}
	= \oint_{\partial S_i} { \left( [N]^T h^3 \frac{\partial \phi}{\partial x} l_x +   [N]^T h^3 \frac{\partial \phi}{\partial z} l_z \right)  d \left( \partial S_i \right)}.
\end{equation*}

Выражение~\eqref{init_eq} примет вид:
\begin{equation*}
	\oint_{\partial S_i} { \left( [N]^T h^3 \frac{\partial \phi}{\partial x} l_x +   [N]^T h^3 \frac{\partial \phi}{\partial z} l_z \right)  d \left( \partial S_i \right)} -
\end{equation*}
\begin{equation*}
	- \int_{S_i} {\left( \frac{\partial[N]^T}{\partial x} \left(h^3 \frac{\partial \phi}{\partial x} \right) +  \frac{\partial[N]^T}{\partial z} \left(h^3 \frac{\partial \phi}{\partial z} \right) - [N]^T 6 \mu U \frac{\partial h}{\partial x}\right) dxdz} = 0.
\end{equation*}

Поскольку в условии задачи заданы граничные условия, то можно не учитывать интеграл по границе. Узлы на которые влияет этот интеграл будут совпадать с узлами, которые зафиксированы граничными условиями.

Итоговое уравнение~\eqref{init_eq} будет иметь вид:
\begin{equation*}
	\int_{S_i} {\left( \frac{\partial[N]^T}{\partial x} \left(h^3 \frac{\partial \phi}{\partial x} \right) +  \frac{\partial[N]^T}{\partial z} \left(h^3 \frac{\partial \phi}{\partial z} \right) - [N]^T 6 \mu U \frac{\partial h}{\partial x}\right) dxdz} = 0.
\end{equation*}

После вычислений этих интегралов можно получить матричное уравнение в локальных координатах относительно $\phi$ следующего вида:
\begin{equation}
W_i \phi = F_i,
\label{ref_local}
\end{equation}
где $F_i$ - правая часть, полученная из интеграла, не содержащего $\phi$:
\begin{equation*}
	F_i = \int_{S_i} [N]^T\left(6 \mu U \frac{\partial h}{\partial x}\right) dx dz.
\end{equation*}

Для вычисления глобальной матрицы требуется для каждого элемента подставить в вырождение~(\ref{ref_local}) для локальных координат узловые границы области и вычислить их значения. 
Далее нужно объединить все матрицы полученные в одну глобальную. Этот процесс будет описан более подробно позже.
После объединения локальных матриц в глобальную получаем матричное уравнение 
\begin{equation*}
	W \phi = F,
\end{equation*}

Далее заменяем значения в матрице $F$ и матрице $W$ так, чтобы значения в граничных узлах совпадали с граничными условиями.
После этого решением уравнения будут являться искомые значения давления $p$ в узловых точках.

\subsection{Построение метода конечных элементов}
В работе уже был описан принцип решения решения уравнения Рейнольдса с помощью слабой формы Галеркина, однако его программная реализация требует дополнительных пояснений.

Для вычисления нового конечного элемента получаются интегралы согласно слабой формы Галеркина в локальной системе координат. Вычисление интегралов необходимо на каждом шаге, поскольку требуется преобразовать функцию зазора $h$ из глобальных координат в локальные, задавая смещение по оси $x$.

Затем происходит вычисление значений узлов конечных элементов, учет граничных условий и их объединение. Так много действий объединено в один этап, поскольку в ином случае для выполнения всех перечисленных действий требовались бы дополнительные вычислительные затраты и использование дополнительной памяти для хранения информации о структуре полученных узлов. Для иллюстрации будем рассматривать решение для сетки с 2 элементами по вертикали и 2 по горизонтали.

Итак, вычисления происходят следующим образом: вначале рассчитывается самый нижний левый конечный элемент. В качестве результата вычисления одного элемента получаем систему из 4 однородных уравнений. Записываем полученную систему в массив решений и сохраняем данные об индексах двух крайних правых узлах \{2,~3\} и двух верхних \{3,~4\}. Важно, что крайние правый элемент мы будем хранить только для самого последнего вычисленного элемента, а значение индексов верхних узлов для всего ряда конечных элементов. По какой причине это происходит именно так стане яснее на следующих этапах вычисления. Поскольку мы имеем граничные условия по условию задачи, то для граничных узлов имеет смысл сразу их учесть. Граничными узлами для рассматриваемого элемента будут являться узлы \{1,~2,~4\}. Заменим уравнения в системе, соответствующие этим узлам, так, чтобы они удовлетворяли граничным условиям. В программе это реализовано в виде уравнения $nodeValue - boundaryValue = 0$, где $nodeValue$ --- неизвестное значение в узле, а $boundaryValue$ --- граничное значение. 
\begin{figure}[!htbp]
	\center{\includegraphics[width=0.7\textwidth, height=0.5\textwidth]{left-bottom-el.jpg}}
	\caption{Нижний левый конечный элемент на сетке 2 на 2 элементов}
	\label{left-bottom-el}
\end{figure}

Далее в цикле вычисляем все оставшиеся конечные элементы в этом ряду, на выбранной сетке он всего один. Поскольку новых узлов в сетке будет всего два, то необходимо будет использовать те же переменные для обозначения уже используемых узлов сетки. Как раз для этого и было сохранено значение индексов крайне правых элементов. Для осуществления слияния двух узлов разных конченых элементов уравнения для этих узлов складываются. В случае, если общие узлы оказываются граничными, то в сложение уравнений смысла не имеет - они и так отвечают граничным условиям. Добавляем в массив индексов верхних элементов новую пару узлов и сохраняем новые крайние правые узлы аналогично прошлому этапу. Аналогично с прошлым этапом учитываем граничные условия
\begin{figure}[!htbp]
	\center{\includegraphics[width=0.7\textwidth, height=0.5\textwidth]{right-bottom-el.jpg}}
	\caption{Нижний правый конечный элемент на сетке 2 на 2 элементов}
	\label{right-bottom-el}
\end{figure}
\newpage
После этого переходим к вычислению конечных элементов выше по оси $z$.  Для этого будет использоваться цикл по вертикальной оси $z$. В рассматриваемом случае в этом цикле будет всего один ряд элементов, лежащих между 0.003 и 0.005 по оси $z$.
Рассмотрим вычисление элемента с узлами \{4,~3,~7,~8\}. Для него большая часть действий аналогично, не будем их повторять. Главной отличительной чертой алгоритма относительно прошлых конечных элементов будет использования массива индексов верхних узлов, который был заполнен на ранних этапах. Он нужен добавления в систему решения общих нижних узлов элементов - \{4,~3\}. Добавление происходит путем складывания имеющегося уравнения в системе с новым уравнением для вычисляемого узла. После все вычислений мы как и ранее сохраняем значение крайне правых узлов и обновляем индекс для крайне верхних узлов в массиве. Стоит заметить, что массив индексов крайне верхних узлов имеет размерность равную числу горизонтальных элементов сетки, то есть размерность 2 для рассматриваемого случае. В нем хранятся только актуальные для вычисления индексы.
\begin{figure}[!htbp]
	\center{\includegraphics[width=0.7\textwidth, height=0.5\textwidth]{left-top-el.jpg}}
	\caption{Верхний левый конечный элемент на сетке 2 на 2 элементов}
	\label{left-top-el}
\end{figure}
\newpage
Для получения последующих уравнений для узлов конченых элементов используется цикл по оси $x$. Каких-то новых приемов для его вычисления не используется, расчет ведется аналогично с прошлыми элементами. В рассматриваемом случае такой элемент всего один.
\begin{figure}[!htbp]
	\center{\includegraphics[width=0.7\textwidth, height=0.5\textwidth]{right-top-el.jpg}}
	\caption{Верхний правый конечный элемент на сетке 2 на 2 элементов}
	\label{right-top-el}
\end{figure}

Итого получаем систему линейных алгебраических уравнений, которая позволит получить нам давления в каждом из узлов сетки. Решаем систему с помощью встроенной в Wolfram Mathematica функции Solve и получаем решение задачи -- набор значений давлений в заданных узлах сетки.

\subsection{Верификация программы на основе обратной подстановки}
В качестве первого способа проверки вначале зададим функцию решения, вычислим для нее правую часть и затем подставим полученную правую часть в нашу программу. Так мы сможем сравнить полученное решение со значениями функции заданной нами самими.

В качестве первой проверочной функции рассмотрим
\begin{equation}
	f(x, z) = -2 \frac{\pi }{0.005} \sin{\left(\frac{\pi z}{0.005}\right)} x \left(x - 0.005\right)
	\label{check_func_1}
\end{equation}
\begin{figure}[!htbp]
	\center{\includegraphics[width=0.9\textwidth, height=0.7\textwidth]{check_func_1.png}}
	\caption{Первая проверочная функция~\eqref{check_func_1}}
	\label{check_func_1_pic}
\end{figure}
Далее вычислим правую часть подставив~\eqref{check_func_1} в уравнение
\begin{equation*}
	F = \frac{\partial^2 f}{x^2} + \frac{\partial^2 f}{\partial z}
\end{equation*}
Полученную правую часть подставим в программу и будем искать искомую функцию $f$.

\begin{table}[!htbp]
	\begin{tabular}{|l|l|l|}
		\hline
		\multicolumn{1}{|c|}{Размерность сетки} & \multicolumn{1}{c|}{Разность, Па} & Погрешность, \% \\ \hline
		5 на 5                                  & 0.0002                              & 3.3            \\ \hline
		10 на 10                                & 0.0001                              & 0.8            \\ \hline
		20 на 20                                & 0.0000                              & 0.2            \\ \hline
	\end{tabular}
\end{table}

\noindent Поскольку форма функции весьма простая, то достаточно 10 элементов, чтобы получить решение с погрешность ниже одного процента.

\begin{figure}[!htbp]
	\center{\includegraphics[width=0.7\textwidth, height=0.5\textwidth]{res_check_func_1.png}}
	\caption{Результат вычислений для проверочной функции~\eqref{check_func_1} на сетке 20 на 20}
	\label{res_check_func_1}
\end{figure}


В качестве второй проверочной функции рассмотрим
\begin{equation}
	f(x, z) = -2 \frac{\pi z}{0.005} \sin{\frac{2 \pi z}{0.005}} \sin{\frac{4 \pi x}{0.005}}
	\label{check_func_2}
\end{equation}
\begin{figure}[!htbp]
	\center{\includegraphics[width=0.7\textwidth, height=0.5\textwidth]{check_func_2.png}}
	\caption{Вторая проверочная функция~\eqref{check_func_2}}
	\label{check_func_2_pic}
\end{figure}
Как видно значение функции на границах нулевое.

Аналогичным образом получим правую часть и подставляем ее в программу.

\begin{table}[!htbp]
	\begin{tabular}{|l|l|l|}
		\hline
		\multicolumn{1}{|c|}{Размерность сетки} & \multicolumn{1}{c|}{Разность, Па} & Погрешность, \% \\ \hline
		5 на 5                                  & 0.969                              & 21.31            \\ \hline
		10 на 10                                & 0.260                              & 5.7            \\ \hline
		20 на 20                                & 0.065                              & 1.4            \\ \hline
	\end{tabular}
\end{table}

Погрешность была вычислена по формуле $\underset{i}{\max} | \frac{{x_t}_i - {x_{node}}_i}{{x_t}_i} |$, где ${x_t}_i$ -- значение в узле $i$, полученное из искомой функции, а ${x_{node}}_i$ -- значение в узле $i$, полученное в программе. Максимум вычисляется по всем узловым элементам.

Итого можно делать вывод, что на приведенных примерах программа считает верно, погрешность сокращается с увеличением числа элементов на сетке. Из двух проверок можно видеть, что для первой более простой по форме функции достаточно 10 элементов для получения небольшой погрешности, тогда как для более сложной второй функции нужно не мена сетка 20 элементов.

\begin{figure}[!htbp]
	\center{\includegraphics[width=0.7\textwidth, height=0.5\textwidth]{res_check_func_2.png}}
	\caption{Результат вычислений для проверочной функции~\eqref{check_func_2} на сетке 20 на 20}
	\label{res_check_func_2}
\end{figure}



\subsection{Верификация программы путем сравнения с Wolfram Mathematica}

В качестве еще одной проверки воспользуемся пакетом компьютерной алгебры Wolfram Mathematica.
Решим уравнение Рейнольдса с помощью функции NDSolve.

\begin{figure}[!htbp]
	\center{\includegraphics[width=0.7\textwidth, height=0.5\textwidth]{w_sol.png}}
	\caption{График решения, полученного с помощью Wolfram Mathematica}
	\label{w_sol}
\end{figure}

Сравним решения полученные с помощью собственной реализацией с решением Wolfram Mathematica.

\begin{table}[!htbp]
	\begin{tabular}{|l|l|l|}
		\hline
		\multicolumn{1}{|c|}{Размерность сетки} & \multicolumn{1}{c|}{Разность, Па} & Погрешность, \% \\ \hline
		5 на 5                                  & 1511	                              & 1.69            \\ \hline
		10 на 10                                & 1211                              & 1.19            \\ \hline
		20 на 20                                & 978                              & 0.96            \\ \hline
	\end{tabular}
\end{table}

Можно заметить, что даже на сетке 5 на 5 получаем близкий к численному решению Wolfram Mathematica. Такие результаты говорят о том, что созданная реализация МКЭ в данном примере выдает верный результат

Верификация работы программы была произведена двумя способами, таким образом можно судить о верности работы программы.

\subsection{Результаты моделирования}

Для демонстрации результатов рассмотрим различные функции зазора $h(x)$.  Расчеты будем проводить на сетке 10 на 10 со стандартными параметрами указанными в постановке задачи~\eqref{reinolts-task}.

В начале вычислим давление для постоянного зазора. Наблюдается симметрия по оси $x$.
\begin{figure}[!htbp]
	\center{\includegraphics[width=0.7\textwidth, height=0.5\textwidth]{res_static.png}}
	\caption{График решения уравнения Рейнольдса для h = 0.001 м}
	\label{res_static}
\end{figure}

Затем получим график для увеличивающегося зазора. Наблюдается сдвиг по оси $x$ влево по сравнению с постоянным зазором.
\begin{figure}[!htbp]
	\center{\includegraphics[width=0.7\textwidth, height=0.5\textwidth]{res_pos.png}}
	\caption{График решения уравнения Рейнольдса для $h = 0.15 x + 0.001$ м}
	\label{res_pos}
\end{figure}

Так выглядит давление на пластину для уменьшающегося зазора. Наблюдается сдвиг по оси $x$ вправо по сравнению с постоянным зазором.
\begin{figure}[!htbp]
	\center{\includegraphics[width=0.7\textwidth, height=0.5\textwidth]{res_neg.png}}
	\caption{График решения уравнения Рейнольдса для $h = -0.15 x + 0.001$ м}
	\label{res_neg}
\end{figure}

Чтобы более наглядно увидеть отличия которые вызывает знак перед коэффициентом у $x$ построим графики с одинаковыми граничными условиями на всех гранях равными $100000$ Па.

При отрицательном коэффициенте видимо провал
\begin{figure}[!htbp]
	\center{\includegraphics[width=0.7\textwidth, height=0.5\textwidth]{zero_neg.png}}
	\caption{График для $h = -0.15 x + 0.001$ м с одинаковыми ГУ}
	\label{zero_neg}
\end{figure}

При положительном коэффициенте наблюдаем пик
\begin{figure}[!htbp]
	\center{\includegraphics[width=0.7\textwidth, height=0.5\textwidth]{zero_pos.png}}
	\caption{График для $h = 0.15 x + 0.001$ м с одинаковыми ГУ}
	\label{zero_pos}
\end{figure}

\newpage
\section{Аэроупругая модель}
Для построения аэроупругой модели не достаточно получить решение уравнения Рейнольдса. Для ее получения необходимо дополнить имеющуюся модель пружиной, ограничивающей движение пластины сверху. Закрепим пружину по центру пластины в точке $\{0.0025, 0.0025\}$. Система имеет две степени свободы.

\begin{figure}[!htbp]
	\center{\includegraphics[width=0.7\textwidth, height=0.5\textwidth]{pruzina.png}}
	\caption{Схема модели пластины с пружиной}
	\label{pruzina}
\end{figure}


Необходимо связать изменения положения пластины и сжатия и растяжение пружины. Для этого необходимо записать уравнения Лагранжа. Они были разработаны французским математиком и астрономом Жозефом Луи Лагранжем в конце 18 века. Идея уравнений Лагранжа второго рода заключается в том, что для описания движения системы достаточно знать ее кинетическую энергию и потенциальную энергию, а также уравнения, описывающие связи между различными частями системы. Эти уравнения выражаются в виде дифференциальных уравнений второго порядка, которые обеспечивают полное описание динамики системы.

Для записи уравнений Лагранжа зададим вертикальную координату l и угловую $\psi$.
\begin{figure}[!htbp]
	\center{\includegraphics[width=0.7\textwidth, height=0.5\textwidth]{Polar_coordinate_components.png}}
	\caption{Полярная система координат}
	\label{Polar_coordinate_components}
\end{figure}

Тогда уравнения Лагранжа будут иметь вид
\begin{equation}
	\begin{cases*}
	\frac{d}{dt} \left(\frac{\partial L}{\partial \dot{x}} \right) - \frac{\partial L}{\partial x} = F, \\
	\frac{d}{dt} \left(\frac{\partial L}{\partial \dot{\psi}} \right) - \frac{\partial L}{\partial \psi} = F,
	\end{cases*}
\label{lagrange}
\end{equation}
$F$ --- сила, $M$ --- момент. $L =  T - P$, где $T$ --- кинетическая энергия, а $P$ --- потенциальная.

\noindent Определим значения потенциальной силы
\begin{equation*}
	T = \frac{m \dot{x^2}}{2} + \frac{I \dot{\psi^2}}{2},
\end{equation*}
где $I = \frac{length^4}{12}$ --- момент инерции квадрата.

\noindent Определим значения потенциальной силы
\begin{equation*}
	P = \frac{k {x^2}}{2} +\frac{c {\psi^2}}{2},
\end{equation*}
где $k$ --- коэффициент жёсткости пружины на деформирование по $l$,  $c$ --- коэффициент жёсткости пружины на деформирование по $\psi$.

После подстановки в исходную систему~\eqref{lagrange} получим
\begin{equation}
	\begin{cases*}
		m \ddot{x} + k x = F, \\
		I \ddot{\psi} + c \psi = M,
	\end{cases*}
	\label{lagrange_fin}
\end{equation}

\subsection{Получение равновесного положения системы}

Для получения равновесного положения на первом этапе вычислим координаты пластины, решив уравнение Рейнольдса методом МКЭ. Начальный зазор $h(x) = 0.001$.
Далее поскольку ищем положение равновесия производные по времени в системе~\eqref{lagrange_fin} можно обнулить
\begin{equation}
	\begin{cases*}
		k x = F, \\
		c \psi = M,
	\end{cases*}
\label{temp_lagr}
\end{equation}
Определим силу $F$ и момент $M$ из полученного ранее уравнения Рейнольдса. Для этого потребуется вычислить интегралы по площади конечного элемента. Обозначим $b_i = p_i - p_{ext}$, где $i = 1..4$, $p_{ext} = 100000$ Па --- внешнее давление на пружину. Составим интегралы для силы и момента
\begin{equation*}
	F = \int_{S_i} { \left( N_1 b_1 + N_2 b_2 + N_3 b_3 + N_4 b_4 \right)  dx dz}
\end{equation*}
\begin{equation*}
	M = \int_{S_i} { \left( N_1 b_1 + N_2 b_2 + N_3 b_3 + N_4 b_4 \right) \sqrt{x^2 + y^2}  dx dz}
\end{equation*}
Вычислив их можно получить значения $l$ и $\psi$ из~\eqref{temp_lagr}. Коэффициенты $k$ и $c$ сложно определить экспериментально, поэтому они были подобраны так, чтобы изменение зазора в достаточной степени влияло на положение пружины.

После получения  $l$ и $\psi$ можно определить новый вид функции зазора $h(x) = \tan{\psi} x + l$. Далее снова получается решение уравнения Рейнольдса и процесс повторяется по кругу до тех пор пока функция  $h(x) = \tan{\psi} x + l$ перестанет изменяться. Получить равновесное значения $h(x)$ удалось за 4 повторения, $h(x) = 0.147 x + 0.00046$. Далее следует убедиться в устойчивости данного положения пластины.


\subsection{Проверка устойчивости системы}
Для проверки устойчивости нужно изменять значения $l$ и $\psi$ на 1-2\% и таким образом построить аппроксимацию $F$ и $M$ из системы~\eqref{temp_lagr}. Аппроксимация полиномиальная вида $F = F_0 + a_1 \Delta{l}$, $M = M_0 + b_1 \Delta{\psi}$.
В искомой системе~\eqref{lagrange_fin} $l$ и $\psi$ заменяются на $l_c + \Delta{l}$ и $\psi_c + \Delta{\psi}$ соответственно. $l_c$, $\psi_c$ --- значения координат в равновесии системы. После подстановки момента и силы, а также замены $l$ и $\psi$ коэффициенты отвечающие за равновесность состояния сокращаются. После этого следует произвести еще две замены
\begin{equation*}
	\Delta{l} = l_c \exp{w_1 t}
\end{equation*}
\begin{equation*}
	\Delta{\psi} = \psi_c \exp{w_2 t}
\end{equation*}

После сокращения экспонент получаем два характеристических выражения для $w_1$ и $w_2$.

Из них по критерию Рауса—Гурвица можно определить асимптотическую устойчивость найденных точек.

\textbf{Критерий Рауса—Гурвица.}
Пусть имеем линейное дифференциальное уравнение с постоянными вещественными коэффициентами:
\begin{equation*}
	a_0 y^{(n)} + a_1 y^{(n - 1)} + ... + a_n y  = 0, (a_0, a_1, ..., a_n = const, a_0 > 0).
\end{equation*}
Нулевое решение $y = 0$ уравнения асимптотически устойчиво, если все корни характеристического уравнения 
\begin{equation*}
	а(\lambda) = a_0 \lambda^n + a_1 \lambda^{n - 1} + ... + a_n = 0.
\end{equation*}
имеют отрицательные вещественные части. Для того чтобы это выполнялось необходимо и достаточно, чтобы были положительными все главные диагональные миноры матрицы Гурвица.

В рассматриваемой задаче миноры матрицы Гурвица будут иметь вид $\Delta_1 = a_1$, $\Delta_2 = a_1 a_2 - a_0 a_3$. Для выполнения условия асимптотической устойчивости необходимо и достаточно чтобы $\Delta_1 > 0$, $\Delta_2 > 0$.

\subsection{Численная реализация}
В численной реализации описанного алгоритма есть некоторые особенности, на которые стоит указать.
Во-первых это связь программы решения МКЭ уравнения Рейнольдса и вычисления новой функции зазора $h(x)$ из уравнений Лагранжа. Для вычисления силы и момента, фигурирующих в системе~\eqref{temp_lagr} требуется сохранить все значения индексов узлов поэлементно, а также смещение для перехода из глобальной системы координат в локальную.



\newpage
\section-{Заключение}
В работе представлена модель решения уравнения Рейнольдса для установившегося течения в газовом смазочном слое построенная с помощью метода конечных элементов. В ходе работы получены следующие результаты:
	\begin{enumerate}	
		\item Создана программная реализация метода конечных элементов для решение уравнения Рейнольдса
		\item Проведена верификация с помощью подстановки правой части и Wolfram Mathematica
		\item Получены решения для различных зазоров
		\item Получено положения равновесия для системы дополненной пружиной.
	\end{enumerate}

\newpage
\begin{thebibliography}{2}
\bibitem{petrov_smazka} Петров Н. Гидродинамическая теория смазки, М.: из-во академии наук СССР, 1948. --- 558~с.
\bibitem{slezkin_smazka} Слезкин Н. Динамика вязкой несжимаемой жидкости, М.: из-во техно-теоретической литературы, 1955. --- 521~с.
\bibitem{seligard} Селегринд Л. Примененение метода конечных элементов, М.: из-во МИР, 1979. --- 195~с.
\bibitem{seshu} Seshu P. Textbook of
Finite Element
Analysis, New Dehli: PHI Learning Private Limited, 2012. --- 340~с.
\bibitem{itmo} Григорьев А. Колебания
и виброактивность элементов машин: Учеб. пособие. СПб.: Университет ИТМО, 2016. --- 136~с.

\end{thebibliography}

\newpage

\section-{Приложение А}

Результаты решения уравнения Рейнольдса МКЭ с постоянным зазором.

\end{document} 