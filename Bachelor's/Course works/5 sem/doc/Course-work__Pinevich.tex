
\documentclass[12pt, a4paper]{article}

\usepackage[utf8]{inputenc}
\usepackage[T1]{fontenc}
\usepackage[russian]{babel}
\usepackage[oglav,spisok,boldsect,eqwhole,figwhole,hyperref,hyperprint,remarks,greekit]{./style/fn2kursstyle}
\graphicspath{{./style/}{./figures/}}

\usepackage{multirow}
\usepackage{supertabular}
\usepackage{multicol}
\usepackage{amsmath}
% Параметры титульного листа
\title{Кручение стержня\\ прямоугольного сечения}
\author{В.\,Г.~Пиневич}
\supervisor{А.\,В.~Котович}
\group{ФН2-51Б}
\date{2023}

% Переопределение команды \vec, чтобы векторы печатались полужирным курсивом
\renewcommand{\vec}[1]{\text{\mathversion{bold}${#1}$}}%{\bi{#1}}
\newcommand\thh[1]{\text{\mathversion{bold}${#1}$}}
%Переопределение команды нумерации перечней: точки заменяются на скобки
\renewcommand{\labelenumi}{\theenumi)}
\begin{document}

\maketitle

\tableofcontents



\newpage

\section-{Введение}
Проблема решения задачи о скручивании балки возникает во многих задачах, в частности в строительной механике. Существует большое количество различных методов решения таких задач. Данная работа посвящена изучению двух численных методов решения таких задач, оценке их точности.

\section{Постановка задачи}
Задача данной работы --- сравнить два решения задачи о кручении стержня прямоугольного сечения, полученных энергетическим методом и методом Ритца. Кроме того, требуется выяснить зависимости точности решения в обоих случаях от числа членов ряда и сравнить полученные результаты.

\subsection{Кручение}
Кручением называется такой вид нагружения стержня, при котором
из всех шести внутренних силовых факторов в его поперечных сечениях не равен нулю только крутящий момент~$M_{\text{кр}}$.

Рассмотрим стержень прямоугольного сечения. 
Такой стержень при
закручивании подвержен депланациям («выходят из плоскости»). Другими словами депланация означает, что точки сечения перемещаются вдоль оси стержня
в различных направлениях.
\begin{figure}[!h]
	\centering
	\includegraphics[width=0.35\textwidth]{sterzhenСruchenie}%
	\caption{Кручение стержня прямоугольного сечения}
	\vspace*{-2mm}
	\label{pic1}
\end{figure}

Будем решать задачу о его кручении двумя способами: энергетическим методом (в виде ряда по ортогональной системе функций) и методом Ритца (в виде ряда по степенным функциям).

Совместим ось $z$ с осью кручения, оси $x$ и $y$ расположим произвольно в плоскости поперечного сечения. Задача кручения сходится к поиску функции $\psi$~(\ref{funCruc})~\cite{Temochenko75}. Эта функция должна быть постоянна вдоль границы поперечного сечения, константу можно выбирать произвольно. Мы будем принимать ее равной нулю.
\begin{equation}
	\label{funCruc}
	\Delta \psi = \frac{\partial^2 \psi }{\partial x^2} +
	\frac{\partial^2 \psi }{\partial y^2} = -2G \theta, 
\end{equation}
\noindent где $G$ --- модуль сдвига, $\theta$ --- угол закручивания на единицу времени.


\subsection{Положительные и положительно определенные операторы}
Рассмотрим некоторый симметричный оператор $A$ гильбертова пространства.
\begin{itemize}
	\item [$\bullet$]
Оператор $A$ называется положительным,
если для любого элемента $u$ из области определения оператора
выполняется неравенство
\[
(Au, u) \geqslant 0,
\]
причем знак равенства имеет место только тогда, когда $u = 0$.
\item [$\bullet$]
Оператор $A$ называется положительно определенным, если существует такая положительная постоянная $\gamma^2$, что для любого элемента $u$ из области определения оператора $A$ справедливо неравенство
\[\label{positive_definite_operator_1}
	(Au, u) \geqslant \gamma^2 \|u\|^2
\]
\end{itemize}

Физический смысл понятия положительно определенного
операторов заключается в том, что невозможно сообщить
системе смещение, не затратив на это некоторой энергии~\cite{michilin_smolskiy}. Если же оператор положительный, но не положительно
определенный, то, системе можно придать сколь угодно большое смещение, затратив на это сколь угодно малую энергию

\section{Энергетический метод}
Рассмотрим положительно определенный оператор $A$ в гильбертовом пространстве $H$. Требуется решить уравнение 
\begin{equation}
	\label{craev_task}
	Au = f, f \in H. 
\end{equation}

Пусть $A$ --- положительный оператор в гильбетровом протестантстве. Тогда энергетическим произведение назовем 
\[
	[u, v] = (Au, v),
\]
\noindent где $u$ и $v$ элементы из области определения $D$ оператора $A$. 

Множество $D(A)$ является гильбертовом пространством, навезем его энергетическим пространством $H_A$. Оно также является сепарабельным.

Тогда мы можем свести решение краевой задачи к задаче о поиске минимума функционала.

Если $A$ --- положительный оператор, уравнение~(\ref{craev_task}) можно свести к поиску минимума функционала
\begin{equation}
	\label{craev-min_task}
	F(u) = (Au, u) - 2(u, f).
\end{equation}
Такой метод решения краевой задачи и называют энергетическим.

Чтобы решение задачи~(\ref{craev-min_task}) существовало расширим функционал $F(u)$ на все пространство $H_A$ и будем искать минимум $u_0$ на нем.
Пространство $H_A$ сепарабельно, то в нем найдется полная ортонормированная система $\omega_n$ и решение $u_0$ можно представить в виде
\begin{equation}
	\label{craev_ser}
	u_0 = \sum\limits_{n = 1}^{\infty} (f, \omega_n)\omega_n.
\end{equation}

\section{Метод Ритца}
Пусть $A$~---~положительно определенный оператор в гильбертовом  пространстве $H$. Задача построения обобщенного решения уравнения~(\ref{craev_task}),
как показано выше, равносильна задаче нахождения элемента энергетического
пространства, который реализует минимум функционала~(\ref{craev-min_task})
в энергетическом пространстве.

Выберем последовательность элементов
\begin{equation}\label{Ritz_phi}
	\varphi_1, \varphi_2, \varphi_3, \ldots,
\end{equation}
удовлетворяющих условиям:
\begin{itemize}
	\item [1.] $\varphi_n \in H_A$, для всех $n$;
	\item [2.] при любом $n$ элементы $\varphi_1, \varphi_2,  \ldots, \varphi_n$, линейно независимы;
	\item [3.] последовательность~\eqref{Ritz_phi} полна в $H_A$.
\end{itemize}
Такие элементы будем называть координатные, а последовательность назовем координатной системой.

Построим линейную комбинацию первых $n$ координатных элементов вида
\begin{equation}\label{Ritz_un_sum}
	u_n = \sum_{i = 1}^n a_i \, \varphi_i,
\end{equation}
с некоторыми $a_i$. Подставим $u_n$ вместо $u$ в~(\ref{craev-min_task}), тогда $F(u)$ можно рассматривать как функцию
независимых переменных $a_1, a_2, \ldots, a_n$
\begin{equation}
	\label{f_ritz}
	F(u_n) = \left[\sum_{i = 1}^n a_i \, A\varphi_i,
	\sum_{k = 1}^n a_k \, A\varphi_k \right] -
	2\left(\sum_{k = 1}^n a_k \, \varphi_k, f \right) =
\end{equation}
\[
	= \sum_{i,k = 1}^n [\varphi_i, \varphi_k] \, a_i a_k -
	2\sum_{k = 1}^n (\varphi_k, f) \, a_k
\]
Выберем коэффициенты $a_i$ так, чтобы минимизировать функцию~(\ref{f_ritz}). Она достигает минимума при тех значениях независимых переменных, которые обращают в нуль ее первые производные. В общем случае эти условия не достаточные, а необходимые условия минимума. Однако, используя положительность оператора $A$, можно доказать, что они в данном случае действительно реализуют минимум $F(u_n)$:
\begin{equation}\label{system_zero_ritz}
	\frac{\partial F(u_n)}{\partial a_i} = 0, \qquad i = 1, 2, \ldots, n.
\end{equation}
Поскольку
\[
	\frac{\partial F(u_n)}{\partial a_i} =
	2 \sum_{k = 1}^n [\varphi_k, \varphi_i] \, a_k - 2\,(f, \varphi_i),
\]
то приравняв эти производные нулю, получим систему линейных алгебраических уравнений  Ритца:
\begin{equation}\label{Ritz_F_system}
	\sum_{k = 1}^n [\varphi_k, \varphi_i] \, a_k = (f, \varphi_i) \qquad
	i = 1, 2, \ldots, n.
\end{equation}

Определитель системы~(\ref{Ritz_F_system})
есть определитель Грама линейно независимых элементов
$\varphi_1, \varphi_2, \ldots, \varphi_n$ и потому отличен от нуля. Следовательно,
система уравнений Ритца всегда однозначно разрешима, если оператор $A$ положительный.

Найдя из~\eqref{Ritz_F_system} коэффициенты $a_1, a_2, \ldots, a_n$ и
подставив их в~\eqref{Ritz_un_sum}, получим приближенное решение уравнения~\eqref{craev_task} по Ритцу.

Приближенное решение задачи о минимуме функционала
энергии, получаемое методом Ритца, совпадает с $n$-ой частной
суммой ряда~\eqref{craev_ser}, представляющего точное решение~\cite{Michilin}.

\section{Решение задачи о кручении стержня энергетическим методом}
Рассмотрим задачу~\cite{Michilin} о кручении стержня, основание которого представляет собой   прямоугольник
$0 \leqslant x \leqslant a$, $0 \leqslant y \leqslant b$.
Функция кручения $\psi(x, y)$ удовлетворяет условию $-\Delta \psi = 2 G \theta$~(\ref{funCruc}).

Функция $\psi(x, y)$ обращается в нуль на
сторонах прямоугольника $x= 0$, $x = a$, $y = 0$, $y = b$.

Энергетическое произведение функций $u(x, y)$, $v(x, y)$ выражается формулой
\begin{equation}  
	\label{multUV}
	[\textit{u}, \textit{v}] = - \int\limits_0^a \int\limits_0^b \left(v(x, y) \Delta u(x, y ) \right) dx dy,
\end{equation}
энергетическая норма
\begin{equation}  
	\label{energoNorm}
	\|u\|^2 = - \int\limits_0^a \int\limits_0^b \left(u(x, y) \Delta u(x, y ) \right) dx dy,
\end{equation}
Функции
\begin{equation}  
	\label{phi_mn}
	\varphi_{mn}(x, y) = \sin\frac{m\pi x}{a} \, \sin\frac{n\pi y}{b}, \quad
	m, n = 1, 2, \ldots
\end{equation}

\begin{itemize}
	\item [$\bullet$] непрерывно дифференцируемы сколько угодно раз и
	обращаются в нуль на контуре прямоугольника и потому входят в
	область определения оператора данной задачи;
	\item [$\bullet$] ортогональны по энергии;
	\item [$\bullet$] не нормированы.
\end{itemize}
Докажем ортогональность, для этого заметим, что
\begin{equation}
	\Delta \varphi_{mn}(x, y) = - \pi^2 \left(\frac{m^2}{a^2} + \frac{n^2}{b^2}\right)\varphi_{mn}(x, y).
\end{equation}
Тогда 
\[
	[\varphi_{mn}, \varphi_{rs}] = - \int\limits_0^a \int\limits_0^b \varphi_{mn} \Delta \varphi_{rs} dx dy
\]
\[
  	[\varphi_{mn}, \varphi_{rs}] = \pi^2(\frac{r^2}{a^2} + \frac{s^2}{b^2})
  	\int\limits_0^a \sin \left(\frac{m \pi x}{a} \sin \frac{r \pi x}{a}\right) dx
  	\int\limits_0^a \sin \left(\frac{n \pi y}{b} \sin \frac{s \pi y}{b}\right) dy.
\]
Если $m \neq r$ или $nx \neq s$, то $[\varphi_{mn}, \varphi_{rs}] = 0$. Пологая, что $r = m$ и $s = n$, найдем 
\[
	\| \varphi_{mn} \|^2 = \frac{\pi^2\left(b^2 m^2 + a^2 n^2\right)}{4ab},
\]
следовательно система функций~(\ref{phi_mn}) не нормированная.
Поделим $\varphi_{mn}$ на $\| \varphi_{mn} \|$, получим систему
\begin{equation}  \label{norm_psi_mn}
	\psi_{mn}(x, y) = \frac{2}{\pi} \sqrt{\frac{ab}{b^2 m^2 + a^2 n^2}} \, \sin\frac{m\pi x}{a} \, \sin\frac{n\pi y}{b}.
\end{equation}

По формуле решения в виде ряда~\eqref{craev_ser} функция кручения представляется рядом
\begin{equation}  \label{norm_psi_mn_series}
	\psi(x, y) = \sum_{m, n = 1}^\infty (2G \theta, \psi_{mn}) \, \psi_{mn}(x, y),
\end{equation}
коэффициенты которого равны
\[
(2G \theta, \psi_{mn}) = \frac{4 G \theta}{\pi}\int\limits_0^a \sin\left(\frac{m \pi x}{a}\right) dx \int\limits_0^b
\sin\left(\frac{n \pi y}{b}\right) dy = 
\]
\[
	= \frac{4abG\theta}{\pi^3 mn} \sqrt{\frac{ab}{b^2 m^2 + a^2 n^2}}[1 - (-1)^m][1 - (-1)^n].
\]
Заметим, что коэффициенты ряда~\eqref{norm_psi_mn_series} равны нулю, если
хотя бы одно из чисел $m$ или $n$ четное. В противном случае
\[
(2G \theta, \psi_{mn}) = \frac{4abG\theta}{\pi^3 mn} \sqrt{\frac{ab}{b^2 m^2 + a^2 n^2}},
\]
откуда, соотношение~\eqref{norm_psi_mn_series} примет итоговый вид
\begin{equation} 
	\label{energy_ans}
	\psi(x, y) = \frac{32 a^2 b^2 G\theta}{\pi^4} \!\!\! \sum_{m, n = 1, 3, 5, \ldots}
	\!\!\! \frac{\sin\dfrac{m\pi x}{a} \, \sin\dfrac{n\pi y}{b}}{mn(b^2 m^2 + a^2 n^2)}.
\end{equation}
Вычислим значение функции $\psi$ кручения в середине квадрата со сторонами $a$~$=$~$b$~$=$~$1$.
\begin{equation} 
	\label{psi_energy_middle}
	\begin{aligned}[b]
	\psi\left(\frac{1}{2}, \frac{1}{2}\right) = 0.144G\theta\mbox{, при } n, m = 1, 3, \\
	\psi\left(\frac{1}{2}, \frac{1}{2}\right) = 0.147G\theta\mbox{, при } n, m = 1, 3, \ldots21,\\
	\psi\left(\frac{1}{2}, \frac{1}{2}\right) = 0.147G\theta\mbox{, при } n, m = 1, 3, \ldots209.
	\end{aligned}
\end{equation}
Получаем, что значение функции кручения c $n = 21$ и $n = 209$ не отличаются по крайней мере до третьего знака, тогда как значение этой функции при $n = 3$ отличается на 3\%.

Рассмотрим график~(\ref{ser_graph}) и линии уровня~(\ref{ser_levels}) полученной функции кручения для $n$~$=$~$21$.
\begin{figure}[!h]
	\centering
	\includegraphics[width=0.6\textwidth]{ser_graph}%
	\caption{График функции кручения, полученной энергетическим способом}
	\vspace*{-2mm}
	\label{ser_graph}
\end{figure}
\begin{figure}[!h]
	\centering
	\includegraphics[width=0.7\textwidth]{ser_levels}%
	\caption{Линии уровня функции кручения, полученной энергетическим способом}
	\vspace*{-2mm}
	\label{ser_levels}
\end{figure}
\newpage
\section{Решение задачи о кручении стержня методом Ритца}
Решение задачи кручения стержня прямоугольного сечения~\cite{Michilin}, как уже было показано выше,  сводится к интегрированию уравнения Пуассона~(\ref{funCruc})
 \[
-\Delta \psi = 2G \theta,
\]
где $G$~--- модуль сдвига, $\theta$~--- угол закручивания стержня на единицу
его длины, в прямоугольнике
\[
0 \leqslant x \leqslant a, 0 \leqslant y \leqslant b
\]
при краевых условиях
\[
\psi(0, y) = \psi(a, y) = \psi(x, 0) = \psi(x, b) = 0.
\]
Полагая для упрощения $\psi = 2G \theta u$, получим задачу в виде
\[
	\begin{cases}
	-\Delta u = 1, \\
	\psi(a, y) =  \psi(x,b) = 0, \\
	\psi(0, y) =  \psi(x,0) = 0.
	\end{cases}
\]

Применим метод Ритца, взяв за координатные функции полиномы.
Рассмотрим многочлен, равный нулю на контуре прямоугольника, т.\,е. на прямых $x = 0$, $x = a$  и $y = 0$, $y = b$ имеют вид
\[
	\psi(x, y) = x(x - a)y(y - b)\left(a_0 + a_1\left(x + y\right) + a_2\left(x^2 + a_3 y^2\right) + \ldots\right).
\]
Для начала ограничимся первым членом ряда. Имеем
\begin{equation}\label{phi0_poly_Ritz}
	\psi_1(x, y) = a_0 x(x - a)y(y - b).
\end{equation}
Вычислим $a_0$.
Для этого подставим функцию $\psi_0(x, y)$~\eqref{phi0_poly_Ritz} в интеграл 
\begin{equation}\label{phi0_int}
	F = - \int\limits_0^a \int\limits_0^b \left(
	\frac{1}{2}
	\left(\frac{\partial\psi}{\partial x}\right)^2 + 
	\frac{1}{2}
	\left(\frac{\partial\psi}{\partial y}\right)^2
	-2G\theta\psi
	 \right) dx dy.
\end{equation}
И из условия~\eqref{system_zero_ritz}
получим
\[
	a_0 = \frac{5 G \theta}{2(a^2 + b^2)}.
\]
Вычислим значение функции $\psi_1$ кручения в середине квадрата со сторонами $a$~$=$~$b$~$=$~$1$
\[
	\psi_1\left(\frac{1}{2}, \frac{1}{2}\right) \approx 0.156 G \theta.
\]
В качестве второго приближения рассмотрим функцию
\begin{equation}\label{u_3_poly_Ritz}
	\psi_2(x, y) = x(x - a)y(y - b)\left(a_0 + a_1\left(x + y\right) + a_2\left(x^2 + a_3 y^2\right)\right).
\end{equation}
Вычислим значение функции $\psi_2$ кручения в середине квадрата, предварительно аналогично первому приближения получив значения $a_0, a_1, a_2$.
\[
\psi_2\left(\frac{1}{2}, \frac{1}{2}\right) \approx 0.146 G \theta.
\]
Получаем, что значение функции кручения второго приближения отличается от значения функции кручения первого приближения в той же точки на 7\%.

Рассмотрим график~(\ref{ritz_graph}) и линии уровня~(\ref{ritz_levels}) полученной функции кручения для первого приближения.
\begin{figure}[!h]
	\centering
	\includegraphics[width=0.7\textwidth]{ritz_graph}%
	\caption{График функции кручения, полученной методом Ритца}
	\vspace*{-2mm}
	\label{ritz_graph}
\end{figure}
\begin{figure}[!h]
	\centering
	\includegraphics[width=0.7\textwidth]{ritz_levels}%
	\caption{Линии уровня функции кручения, полученной методом Ритца}
	\vspace*{-2mm}
	\label{ritz_levels}
\end{figure}
\newpage
\section{Сравнение решений энергетическим методом и методом Ритца}

Сравним два рассматриваемых метода, сопоставив результаты полученных функций кручения, а так же их крутящих моментов.

\subsection{Сравнение функций кручения}
Мы получили	значение $\psi\left(\frac{1}{2}, \frac{1}{2}\right) = 0.147G\theta$ для энергетического метода и $\psi\left(\frac{1}{2}, \frac{1}{2}\right) = 0.146G\theta$ для метода Ритца, они отличаются на 1\%. При этом при малой точности вычисления данные методы дают результат $0.144G\theta$ и $0.156G\theta$ соответственно. Точным решением является $\psi\left(\frac{1}{2}, \frac{1}{2}\right) = 0.147G\theta$~\cite{Temochenko75}, из чего мы можем сделать вывод, что метод энергий более точный в рассматриваемом случае.
\begin{figure}[!h]
	\centering
	\includegraphics[width=1\textwidth]{compare3d}%
	\caption{
		Сравнение функций кручения, полученных энергетическим методом и методом Ритца
	}
	\vspace*{-2mm}
	\label{ritz_graph}
\end{figure}
\newpage
\subsection{Сравнение крутящего момента}
Крутящий момент определяется формулой
\begin{equation}
	\label{torque_com}
	M = 2 \int\limits_{-a}^a \int\limits_{-b}^b \varphi dx dy.
\end{equation}

Для метода энергий, интегрируя ряд~\eqref{energy_ans}, получим
\begin{equation}\label{torque_Miclin}
	M_1 = \frac{256 G \theta a^4 b^4} {\pi^6}\!\!\!\sum_{m, n = 1, 3, 5, \ldots}\!\!\!\frac{1}{(b^2 m^2 + a^2 n^2 )m^2 n^2}.
\end{equation}
Преобразуем ряд~\eqref{torque_Miclin}, получим
\begin{equation}\label{torque_Miclin_th}
M_1 = \frac{1}{3} \, G \theta a^3 b \left( 1 - \frac{192 a}{\pi^5 b}\!\!\!\sum_{n = 1, 3, 5, \ldots}\!\!\!\frac{1} {n^5} \th\dfrac{n \pi b} {2 a } \right).
\end{equation}
Приняв, что прямоугольник достаточно узкий можем положить $\th\dfrac{n \pi b} {2 a } = 1$  и представить  выражение~\eqref{torque_Miclin_th} в виде
\[
M_1=\frac{1}{3}\left(1 - 0.630\frac{a}{b}\right).
\]
Для квадрата получим
\[
M_1=0.140G\theta a^4.
\]

C другой стороны, для метода Ритца первого приближения получаем
\[
M_2 = 0.139G\theta a^4,
\]
а для второго приближения
\[
M_2 = 0.140G\theta a^4.
\]
Таким образом, при рассмотрении достаточно узкого прямоугольника крутящий момент, полученный энергетическим методом можно считать за точный, а метод Ритца будет совпадать с точным решением до третьего знака при расчете во втором приближении.
\newpage
\section-{Заключение}
В ходе выполнения курсовой были изучены энергетический метод и метод Ритца нахождения кручения стержня прямоугольного сечения.	
С помощью этих методов была решена задача, их результаты оказались идентичны с точностью до двух знаков. Оба метода позволяют вычислить крутящий момент с точностью до третьего знака. Энергетический метод позволяет получить ответ  несколько точнее и быстрее, но каждый раз требует вычисления тригонометрического ряда. Метод Ритца дает возможность получить функцию кручения в виде многочлена и получать ответ с другими параметрами задачи с меньшим количеством вычислений, что может быть полезно при большом объеме вычислений. 
\newpage
\begin{thebibliography}{2}
\bibitem{Michilin} С. Г. Михлин. Вариационные методы в математической физике, М.: Изд-во Наука, 1970. --- 512~с.

\bibitem{Temochenko75} С. П. Тимошенко, Дж. Гудьер. Теория упругости, М.: Изд-во Наука, 1975. --- 576~с.

\bibitem{michilin_smolskiy} С. Г. Михлин, Х.Л. Смолицкий Приближенные методы решения дифференциальных и интегральных уравнений. М.: Изд-во Наука, 1965. --- 384~с.

\end{thebibliography}

\end{document} 